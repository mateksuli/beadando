\startproject iras
\mainlanguage[hu]
\setupalign[
  justified,
  nothanging,
  nohz,
  hyphenated,
  morehyphenated,
  tolerant,
]
\setupinterlinespace[height=0.75,depth=0.25]
\setuplayout[
  grid=no,
  location=middle,
]
\setupformulas[align=middle]
%\setupmathalignment[grid=no]

\def\PagenumberingCommand#1{\doifnot\pagenumber1{#1}}
\setuppagenumbering[
  location={footer,middle},
  command=\PagenumberingCommand,
]
\setuppapersize[A4]
\setuplayout[
    backspace=30mm,
    width=150mm,
    topspace=30mm,
    header=0mm,
    footer=15mm,
    footerdistance=0mm,
    bottom=0mm,
    bottomdistance=0mm,
    height=247mm
]

% Betűkészlet
%\setupbodyfont[libertinus,12pt]
\setupbodyfont[12pt]

% Vékony spácium bizonyos karakterek előtt (:;?!)
\definecharacterspacing [magyarpunctuation]
\setupcharacterspacing [magyarpunctuation] ["0021] [left=.1,alternative=1] % ! % strip preceding space(char)
\setupcharacterspacing [magyarpunctuation] ["003A] [left=.1,alternative=1] % : % strip preceding space(char)
\setupcharacterspacing [magyarpunctuation] ["003B] [left=.1,alternative=1] % ; % strip preceding space(char)
\setupcharacterspacing [magyarpunctuation] ["003F] [left=.1,alternative=1] % ? % strip preceding space(char)

% A magyar nyelv beállításai
\startsetups[magyar]
  % Vékony spácium bizonyos karakterek előtt (:;?!)
  \setcharacterspacing[magyarpunctuation]
  \setupindenting[%
    yes,% A bekezdéseket behúzással kezdjük.
    %next,% Az első bekezdés nincs behúzva.
    2em% Közepes méretű (átmeneti megoldás: igazából a mérete 24 cicerós sorig 1 kvirt, nagyobbbnál 2 kvirt kellene legyen -> TENNIALÓ)
  ]
\stopsetups

\setuplanguage[hu][%
  setups=magyar,% Érvényesíti a fent megadott beállításokat.
  spacing=packed% Frenchspacing (Gyurgyák 319. o.: egyenletes szóközök).
                % http://wiki.contextgarden.net/French_spacing).
]

% TENNIVALÓ: csak magyar nyelvre
% Idézetek (Gyurgyák, 86--87. o.).
\definedelimitedtext[quote][location=text]
\setupdelimitedtext[quote:1][
  left={\lowerleftdoubleninequote},
  right={\upperrightdoubleninequote},
  spaceafter=0
]
\setupdelimitedtext[quote:2][
  left={\rightguillemot\nobreak\hskip-.07em},
  right={\kern-0.03em\leftguillemot},
  spaceafter=0
]
\setupdelimitedtext[quote:3][
  left={\upperleftsingleninequote},
  right={\upperrightsingleninequote},
  spaceafter=0
]

\definebodyfontenvironment[default][em=italic]

\defineframedtext[kerdes][align=center,offset=0.5ex,style=italic,width=\dimexpr0.8\dimexpr\makeupwidth]

\defineframedtext[kivonat][offset=0.5ex, frame=off,style=italic,width=\dimexpr0.8\dimexpr\makeupwidth]

\definehead[cim][chapter]
\setuphead[cim][number=no,align=middle,after={},]

\define[2]\sectioncommand{\hbox{#1. #2}}
\setuphead[section][
  sectionsegments=section,
  command=\sectioncommand,
]

\define[2]\feladatcommand{\hbox{Feladat #1 \enskip #2}}
\definehead[feladat][subsection]
\setuphead[feladat][
    number=yes,
    textdistance=0pt,
    alternative=text,
    style=bf,
    commandafter={\enskip},
    command=\feladatcommand,
    sectionsegments=3:100,
    indentnext=no,
    beforesection={\setupindenting[no]},
    aftersection={\setupindenting[yes]},
]

\define[1]\megoldascommand{\hbox{Megoldás #1}}
\definehead[megoldas][subsubject]
\setuphead[megoldas][
    number=no,
    textdistance=0pt,
    alternative=text,
    style=bf,
    commandafter={\quad},
    command=\megoldascommand,
    sectionsegments=3:100,
    indentnext=no,
]

\define[1]\rovidpontmegjegyzes{\setupinterlinespace[1ex]\framed[align=flushright,width=\textwidth,offset=0pt,frame=off]{\framed[location=middle,align={end},offset=0pt,width=0.618\textwidth,frame=off]{\small #1}}\setupinterlinespace[2.8ex]} % 0,618 az aranymetszés; muszáj volt a 2.8 ex-et beírni, reset és paraméter néllkül nem jó

\define[1]\hosszupontmegjegyzes{\setupinterlinespace[1ex]\framed[align=flushright,width=\textwidth,offset=0pt,frame=off]{\framed[location=middle,align={normal},offset=0pt,width=0.618\textwidth,frame=off]{\small #1}}\setupinterlinespace[2.8ex]} % 0,618 az aranymetszés; muszáj volt a 2.8 ex-et beírni, reset és paraméter néllkül nem jó

\define[1]\megjegyzes{\setupinterlinespace[1ex]\framed[location=middle,align={normal},width=\textwidth,frame=off,offset=0pt,toffset=1ex]{\small #1}\setupinterlinespace[2.8ex]}

\definebar[mathubar][underbar]  % enélkül esetleg elcsúszik az aláhúzás

%\showgrid[all]



% https://tex.stackexchange.com/a/459615/50554
\def\feladatpont#1{{\unskip\nobreak\hfil\penalty50
  \hskip2em\hbox{}\nobreak\hfil #1\/ 
  \parfillskip=0pt \finalhyphendemerits=0 \par}}

\defineitemgroup[alfeladat]
\setupitemgroup[alfeladat][each][right=),margin=2em,before={},inbetween={},after={},stopper=,style=italic]

\startproduct iras

\useURL[segedanyag][https://github.com/mateksuli/segedanyag]

\startbodymatter

\cim{Beadandó dolgozat megoldások}
\startlinealignment[middle]
ötödik osztály, 2019. április, {5/12}
\stoplinealignment

\blank[2*big]

\startsection[title=Összeadás és kivonás]

\startmegoldas[title=1.2] \feladatpont{$2\cdot5=10$ pont}

\rovidpontmegjegyzes{Forrás: Imrecze et al{.}: {\em Fejtörő feladatok felsősöknek}\\(III.2.17)}

\noindent 770 és 77.
Ha a kisebbik szám $x$, akkor a nagyobb $10x$. Márpedig $x+10x=11x$, és így $11x=847$, amiből $x=77$.

\megjegyzes{A feladat rendesen valószínűleg kevesebbet érne, de a cél a magasabb pontszámmal most az, hogy motiváljon arra, hogy foglalkozzanak vele, ami viszont előkészíti az egyenlettel való megoldás és így az egyenletek bemutatását.}
\stopmegoldas

\startmegoldas[title=1.3] \feladatpont{2 pont/db}

\noindent {\starttabulate[|Mcw(\dimexpr0.5\textwidth-0.5em\relax)|Mcw(\dimexpr0.5\textwidth-0.5em\relax)|]
 \NC 317554+61531=379085 \NC 94765+73582=168347\NR 
 \NC 28777+68531=97308 \NC 35118+7781=42899\NR 
 \NC 98621+602043=700664 \NC 99602+18492=118094\NR 
 \NC 15775+2524=18299 \NC 30531+9328=39859\NR 
 \NC 6228+16556=22784 \NC 443550+1824=445374\NR 
 \NC 732201-3230=728971 \NC 63088-30729=32359\NR 
 \NC 6590-6210=380 \NC 10601-9330=1271\NR 
 \NC 723790-92424=631366 \NC 89164-4137=85027\NR 
 \NC 183084-91390=91694 \NC 630265-84187=546078\NR 
 \NC 6283-2833=3450 \NC 89768-74068=15700\NR 
\stoptabulate
}  
\stopmegoldas

\startmegoldas[title=1.4] \feladatpont{2 pont/db}

\noindent {\starttabulate[|Mcw(\dimexpr0.5\textwidth-0.5em\relax)|Mcw(\dimexpr0.5\textwidth-0.5em\relax)|]
 \NC (-82)-(-25)+(-21)=-78 \NC 6+(-517)+21=-490\NR 
 \NC (-29)+414+95=480 \NC 298+30-143=185\NR 
 \NC 90+(-86)+(-4)=0 \NC (-1)-(-94)-(-72)=165\NR 
 \NC 71-245-0=-174 \NC 2-16-(-526)=512\NR 
 \NC 0-57-34=-91 \NC (-5)+12-69=-62\NR 
\stoptabulate
}  
\stopmegoldas

\startmegoldas[title=1.5] \feladatpont{$5\cdot2=10$ pont}

\rovidpontmegjegyzes{Forrás: Imrecze et al{.}: {\em Fejtörő feladatok felsősöknek}\\(III.2.9)}

\noindent Mivel:
\startformula \startmathalignment[n=5,align={left,middle,left,middle,left}]
\NC \text{\bf B}+\text{\bf B}<19\text{,}\quad\NC\text{azért}\NC\quad\text{\bf A}=1\text{;} \NR
\NC \text{\bf E}+\text{\bf E}=\text{\bf E}\text{,}\quad\NC\text{azért}\NC\quad\text{\bf E}=0\text{;} \NR
\NC \text{\bf E}=0\text{,}\quad\NC\text{azért}\NC\quad\text{\bf B}+\text{\bf B}=10\quad\NC\text{és}\NC\quad\text{\bf B}=5\text{;} \NR
\NC \text{\bf C}+\text{\bf A}=5\text{,}\quad\NC\text{azért}\NC\quad\text{\bf C}=4\text{;} \NR
\NC \text{\bf C}=4\text{,}\quad\NC\text{azért}\NC\quad\text{\bf D}+\text{\bf D}=4\quad\NC\text{és}\NC\quad\text{\bf D}=2\text{.} \NR
\stopmathalignment \stopformula
Az összeadás tehát:
\startformula
{\framed[frame=off,align={flushright,nothyphenated,verytolerant},before={},after={}]{
$5240$\\
\underbar{$+5210$}\\
$10450$\\
}}
\stopformula
\stopmegoldas

\startmegoldas[title=1.6] \feladatpont{20 pont}

\noindent A megoldás 5050.
Egy Gaussról szóló híres történet, amely a szájhagyomány útján átalakult, arról szól, hogy Gauss általános iskolai tanára, J{.} G{.} Büttner diákjait azzal akarta lefoglalni, hogy 1-től 100-ig adják össze az egész számokat.
A fiatal Gauss mindenki megdöbbenésére másodpercek alatt előrukkolt a helyes megoldással, megvillantva matematikai éleselméjűségét: a számsor alá visszafele leírta a számokat, majd az oszlopokat összeadta, így azonos összegeket kapott:
\startformula
1 + 100 = 2 + 99 = 3 + 98 = \dots = 50 + 51= 101.
\stopformula
Ez összesen 50 darab számpárt jelentett, és így $50\cdot101=5050$.

\megjegyzes{A 20 pont a fáradságos munkát vagy a találékonyságot hivatott díjazni, valamint motivál a feladat elvégzésére és ezzel megágyaz a fenti történet és egyúttal a számtani sorok bemutatásának.}
\stopmegoldas

\startmegoldas[title=1.7] \feladatpont{10 pont}

\noindent $\text{A}=1$, 
$\text{Á}=2$, 
$\text{B}=3$, 
$\text{C}=4$, 
$\text{Cs}=5$, 
$\text{D}=6$, 
$\text{Dz}=7$, 
$\text{Dzs}=8$, 
$\text{E}=9$, 
$\text{É}=10$, 
$\text{F}=11$, 
$\text{G}=12$, 
$\text{Gy}=13$, 
$\text{H}=14$, 
$\text{I}=15$, 
$\text{Í}=16$, 
$\text{J}=17$, 
$\text{K}=18$, 
$\text{L}=19$, 
$\text{Ly}=20$, 
$\text{M}=21$, 
$\text{N}=22$, 
$\text{Ny}=23$, 
$\text{O}=24$, 
$\text{Ó}=25$, 
$\text{Ö}=26$, 
$\text{Ő}=27$, 
$\text{P}=28$, 
$\text{Q}=29$, 
$\text{R}=30$, 
$\text{S}=31$, 
$\text{Sz}=32$, 
$\text{T}=33$, 
$\text{Ty}=34$, 
$\text{U}=35$, 
$\text{Ú}=36$, 
$\text{Ü}=37$, 
$\text{Ű}=38$, 
$\text{V}=39$, 
$\text{W}=40$, 
$\text{X}=41$, 
$\text{Y}=42$, 
$\text{Z}=43$, 
$\text{Zs}=44$.

\megjegyzes{A feladat burkoltan az is, hogy megtanuljuk, hogy mi a különbség a 40 betűs magyar ábécé és a 44 betűs kiterjesztett magyar ábécé, ami tartalmazza a Q, W, X, Y betűket is. Az interneten a ,,magyar ábécé'' keresőszó segítségével nyerhet az ember felvilágosítást erről. Emmellet a adatok kódolásába is bevezet.}
\stopmegoldas



\startsection[title=Prímszámok, prímtényezőkre bontás]

\startmegoldas[title=2.2] \feladatpont{$4+4=8$ pont}
\rovidpontmegjegyzes{Forrás: Imrecze et al{.}: {\em Fejtörő feladatok felsősöknek}\\(III.1.10)}

Sárinak igaza volt és a kockák felső lapján 3, 3, és 5 vagy 5, 5 és 3 pötty lehetett.

A három kockával legfeljebb $3\cdot6=18$-at lehet dobni. Ennél kisebb, de 10-nél nagyobb törzsszámok: 11 és 13. A 11 három törzsszám összegeként csak mint $3+3+5$ kapható, a 13 pedig mint $5+5+3$. Más megoldás nincs.
\stopmegoldas

\startmegoldas[title=2.3] \feladatpont{20 pont}

\noindent {\starttabulate[|Mcw(\dimexpr0.166\textwidth-0.8300000000000001em\relax)|Mcw(\dimexpr0.167\textwidth-0.8350000000000001em\relax)|Mcw(\dimexpr0.167\textwidth-0.8350000000000001em\relax)|Mcw(\dimexpr0.167\textwidth-0.8350000000000001em\relax)|Mcw(\dimexpr0.167\textwidth-0.8350000000000001em\relax)|Mcw(\dimexpr0.166\textwidth-0.8300000000000001em\relax)|]
 \NC 2 \NC 3 \NC 5 \NC 7 \NC 11 \NC 13\NR 
 \NC 17 \NC 19 \NC 23 \NC 29 \NC 31 \NC 37\NR 
 \NC 41 \NC 43 \NC 47 \NC 53 \NC 59 \NC 61\NR 
 \NC 67 \NC 71 \NC 73 \NC 79 \NC 83 \NC 89\NR 
 \NC 97 \NC 101 \NC 103 \NC 107 \NC 109 \NC 113\NR 
 \NC 127 \NC 131 \NC 137 \NC 139 \NC 149 \NC 151\NR 
 \NC 157 \NC 163 \NC 167 \NC 173 \NC 179 \NC 181\NR 
 \NC 191 \NC 193 \NC 197 \NC 199\NR 
\stoptabulate
}
\stopmegoldas

\startmegoldas[title=2.4] \feladatpont{12 pont}
\rovidpontmegjegyzes{Forrás: Imrecze et al{.}: {\em Fejtörő feladatok felsősöknek}\\(III.1.5)}

\noindent Az ötjegyű szám: 23572.
Az egyjegyű törzsszámok: 2, 3, 5, 7 összege 17, és ha közülük a 2-t adjuk a 17-hez, csak akkor jutunk törzsszámhoz.
Az {\bf MA} csak 23 lehet, mert sem 25, sem 27 nem prímszám. Tehát $\text{\bf M}=2$, $\text{\bf A}=3$.
Az {\bf MLO} vagy 257 vagy 275 lehetne, de 275 nem törzsszám, ezért $\text{\bf L}=5$, $\text{\bf O}=7$. 
\stopmegoldas


\startmegoldas[title=2.5] \feladatpont{2 pont/db}

\noindent {\starttabulate[|Mcw(\dimexpr0.5\textwidth-0.5em\relax)|Mcw(\dimexpr0.5\textwidth-0.5em\relax)|]
 \NC 124=2^2\cdot31 \NC 280=2^3\cdot5\cdot7\NR 
 \NC 132=2^2\cdot3\cdot11 \NC 24=2^3\cdot3\NR 
 \NC 88=2^3\cdot11 \NC 1672=2^3\cdot11\cdot19\NR 
 \NC 120=2^3\cdot3\cdot5 \NC 2002=2\cdot7\cdot11\cdot13\NR 
 \NC 96=2^5\cdot3 \NC 10472=2^3\cdot7\cdot11\cdot17\NR 
 \NC 64=2^6 \NC 80=2^4\cdot5\NR 
 \NC 72=2^3\cdot3^2 \NC 3420=2^2\cdot3^2\cdot5\cdot19\NR 
 \NC 168=2^3\cdot3\cdot7 \NC 48=2^4\cdot3\NR 
 \NC 84=2^2\cdot3\cdot7 \NC 30=2\cdot3\cdot5\NR 
 \NC 1815=3\cdot5\cdot11^2 \NC 8=2^3\NR 
\stoptabulate
}  
\stopmegoldas


\startmegoldas[title=2.6]

\rovidpontmegjegyzes{Forrás: Imrecze et al{.}: {\em Fejtörő feladatok felsősöknek}\\(III.1.27)}

\noindent Egy kész törzszsámtáblázat segítségével (lásd Feladat 2.3) könnyen megoldhatjuk a feladatot -- azonban e nélkül is célhoz érhetünk.

Az {\it a)} kérdésre a választ megtaláljuk a {\em Prímszámok} segédanyag {\em Eratosztenész szitája} című fejezetében. De a következőképpen is gondolkodhatunk: jelöljük $n$-nel a $2\cdot3\cdot5\cdot7\cdot11$ számot, ekkor az $n+2$, $n+3$, $n+4$, \dots, $n+11$ tíz egymást követő szám között nincs prímszám, hiszen mindegyiknek van 1-nél nagyobb és nála kisebb osztója, mert mindegyik osztható a 2, 3, 5, 7, 11 törzsszámok valamelyikével. A prímszámtáblázatból azt olvashatjuk ki, hogy először a 114-gyel kezdődő 10 (13) egymást követő szám között nem találunk prímet.
\feladatpont{3 pont}
\startalfeladat[a][start=2]
\item Például: $48, 49, 50, 51, 52, \mathubar{53}, 54, 55, 56, 57$;
\feladatpont{3 pont}
\item Például: $\mathubar{19}, 20, 21, 22, \mathubar{23}, 24, 25, 26, 57, 28$;
\feladatpont{3 pont}
\item Például: $\mathubar{7}, 8, 9, 10, \mathubar{11}, 12, \mathubar{13}, 14, 15, 16$;
\feladatpont{3 pont}
\item Például: $\mathubar{3}, 4, \mathubar{5}, 6, \mathubar{7}, 8, 9, 10, \mathubar{11}, 12$.
\feladatpont{3 pont}
\stopalfeladat

\noindent Öt prímszám is előfordulhat tíz egymást követő szám között: $\mathubar{2}, \mathubar{3}, 4, \mathubar{5}, 6, \mathubar{7}, 8, 9, 10, \mathubar{11}$, de ennél több nem, hiszen a 2-nél nagyobb páros számok nem törzsszámok és tíz egymást követő szám között öt páros van.
\feladatpont{3 pont}
\stopmegoldas

\emptylines[1]\noindentation Hosszúhetény, 2019{.} április 14.

\stopsubject

\stopbodymatter
\stopproduct
\stopproject
