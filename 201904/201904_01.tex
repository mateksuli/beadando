\startproject iras
\mainlanguage[hu]
\setupalign[
  justified,
  nothanging,
  nohz,
  hyphenated,
  morehyphenated,
  tolerant,
]
\setupinterlinespace[height=0.75,depth=0.25]
\setuplayout[
  grid=no,
  location=middle,
]
\setupformulas[align=middle]
%\setupmathalignment[grid=no]

\def\PagenumberingCommand#1{\doifnot\pagenumber1{#1}}
\setuppagenumbering[
  location={footer,middle},
  command=\PagenumberingCommand,
]
\setuppapersize[A4]
\setuplayout[
    backspace=30mm,
    width=150mm,
    topspace=30mm,
    header=0mm,
    footer=15mm,
    footerdistance=0mm,
    bottom=0mm,
    bottomdistance=0mm,
    height=247mm
]

% Betűkészlet
%\setupbodyfont[libertinus,12pt]
\setupbodyfont[12pt]

% Vékony spácium bizonyos karakterek előtt (:;?!)
\definecharacterspacing [magyarpunctuation]
\setupcharacterspacing [magyarpunctuation] ["0021] [left=.1,alternative=1] % ! % strip preceding space(char)
\setupcharacterspacing [magyarpunctuation] ["003A] [left=.1,alternative=1] % : % strip preceding space(char)
\setupcharacterspacing [magyarpunctuation] ["003B] [left=.1,alternative=1] % ; % strip preceding space(char)
\setupcharacterspacing [magyarpunctuation] ["003F] [left=.1,alternative=1] % ? % strip preceding space(char)

% A magyar nyelv beállításai
\startsetups[magyar]
  % Vékony spácium bizonyos karakterek előtt (:;?!)
  \setcharacterspacing[magyarpunctuation]
  \setupindenting[%
    yes,% A bekezdéseket behúzással kezdjük.
    %next,% Az első bekezdés nincs behúzva.
    medium% Közepes méretű (átmeneti megoldás: igazából a mérete 24 cicerós sorig 1 kvirt, nagyobbbnál 2 kvirt kellene legyen -> TENNIALÓ)
  ]
\stopsetups

\setuplanguage[hu][%
  setups=magyar,% Érvényesíti a fent megadott beállításokat.
  spacing=packed% Frenchspacing (Gyurgyák 319. o.: egyenletes szóközök).
                % http://wiki.contextgarden.net/French_spacing).
]

% TENNIVALÓ: csak magyar nyelvre
% Idézetek (Gyurgyák, 86--87. o.).
\definedelimitedtext[quote][location=text]
\setupdelimitedtext[quote:1][
  left={\lowerleftdoubleninequote},
  right={\upperrightdoubleninequote},
  spaceafter=0
]
\setupdelimitedtext[quote:2][
  left={\rightguillemot\nobreak\hskip-.07em},
  right={\kern-0.03em\leftguillemot},
  spaceafter=0
]
\setupdelimitedtext[quote:3][
  left={\upperleftsingleninequote},
  right={\upperrightsingleninequote},
  spaceafter=0
]

\definebodyfontenvironment[default][em=italic]

\defineframedtext[kerdes][align=center,offset=0.5ex,style=italic,width=\dimexpr0.8\dimexpr\makeupwidth]

\defineframedtext[kivonat][offset=0.5ex, frame=off,style=italic,width=\dimexpr0.8\dimexpr\makeupwidth]

\definehead[cim][chapter]
\setuphead[cim][number=no,align=middle,after={},]

\define[2]\sectioncommand{\hbox{#1. #2}}
\setuphead[section][
  sectionsegments=section,
  command=\sectioncommand,
]

\define[2]\feladatcommand{\hbox{Feladat #1 \enskip #2}}
\definehead[feladat][subsection]
\setuphead[feladat][
    number=yes,
    textdistance=0pt,
    alternative=text,
    style=bf,
    commandafter={\enskip},
    command=\feladatcommand,
    sectionsegments=3:100,
    indentnext=no,
    beforesection={\setupindenting[no]},
    aftersection={\setupindenting[yes]},
]

\definebar[mathubar][underbar]  % enélkül esetleg elcsúszik az aláhúzás

%\showgrid[all]



% https://tex.stackexchange.com/a/459615/50554
\def\feladatpont#1{{\unskip\nobreak\hfil\penalty50
  \hskip2em\hbox{}\nobreak\hfil #1\/ 
  \parfillskip=0pt \finalhyphendemerits=0 \par}}

\startproduct iras

\useURL[segedanyag][https://github.com/mateksuli/segedanyag]

\startbodymatter

\cim{Beadandó dolgozat feladatok}
\startlinealignment[middle]
ötödik osztály, 2019. április\\
{{azonosító}}
\stoplinealignment

\blank[2*big]

\noindentation Kedves tanulók! 
\blank
\noindentation Megpróbáltam nektek elmesélni, hogy a matematika, ami kizárólag a logikus emberi gondolkodás gyermeke, mennyire erős és örök.
Valóban, a matematikát nem lehet gondolkodás nélkül megúszni, csak ha az ember megszokja, hogy itt egy probléma és törnöm kell a fejem rajta, ha meg akarom oldani.
Sokakat ez riaszt, és jobbnak látják, ha azt mondják maguknak, hogy ők ezeket a típusú problémákat nem akarják megoldani, őket ez nem érdekli.
George Bernard Shaw mondta: \language[en] ,,Two percent of the people think; three percent of the people think they think; and ninety-five percent of the people would rather die than think.'' \language[hu]
Én arra buzdítalak titeket, hogy eddzétek az agyatokat, nehogy eltunyuljon.
A matematika ebben segít.
Mint minden edzés, először kellemetlen ez is.
De ha nem adjátok fel, hamarosan beérik és meglepően szép gyümölcsöket fog teremni.
E gyümölcsök leszakításáért persze aztán is meg kell majd dolgozni, de az már olyan lesz, mint egy kellemes tavaszi szüret.

Sokat gondolkodtam azon, hogy milyen formában legyen a számonkérés. 
Úgy érzem, hogy voltak közös óráink amik különösen kellemesen teltek, és olyanok is, amiket jobb elfelejteni mindkét félnek.
A matematika nagyon kevés alaptörvényre épített hatalmas logikai vár, amit élvezettel alkottak olyan emberek akiket ez vonz.
A tanár azon dolgozik, hogy úgy mutassa be a tantárgyát, hogy vonzalmat ébresszen, hiszen amit nem ismerünk azt nem is szerethetjük.
A kellemes óráink inkább azok voltak, amik játékosan teltek és inkább erre irányultak.
Többször előfordult, hogy egy jól eltöltött óra útán tévesen úgy kezeltelek titeket, mintha már egy igazán érdeklődő csapatom volna.
Na ezek az órák sikerültek kellemetlenre.
Igyekszem ezt a hullámzást megszűntetni és arra alapozni, ami működik: a matematika játékos oldalára, illetve arra, hogy senkitől ne kívánjak erején felüli teljesítményt.

Ez a gondolat sarkallt arra, hogy átszabjam ezt a feladatsort is.
A fő motívum az lett, hogy a felkínált feladatok közül szabadon válogathassatok.
Mindenkire rábízom, hogy melyik feladatot dolgozza ki.
A lényeg, hogy 100 pont kell az ötös, 75 a négyes, 50 a hármas, és 25 pont a kettes osztályzathoz.
Minthogy a hibás megoldások csak töredékpontokkal járnak, így aki biztosra akar menni a jegyét illetően, az nyugodtan mehet 100 pont fölé is.

Néhány szabály azonban van!
A dolgozatot külön lapon vagy lapokon kérem beadni.
Az egyed feladatok kidolgozásai és megoldásai jól azonosíthatók legyenek.
Minden esetben tollal dolgozzatok!
A munkalapokon ki ne satírozzatok semmit, egy vagy két vonallal áthúzni viszont természetesen bármit szabad.
Nem baj, ha akár oldalakon keresztül is nekiestek egy-egy feladat megoldásának, százszor is kihúzva a sikertelen próbálkozásokat; minden ilyen lapot adjatok be, a kitartó törekvést értékelni fogom, higyjétek el.
Ha viszont egy ilyen összevisszaság közepén rábukkantok a helyes megoldásra, azt szépen, elhatárolt területen emeljétek ki, akár annak az árán is, hogy még egyszer letisztult formában le kell írjátok az egész levezetést.

Vannak feladatok, melyek bizonyos segédanyagok tanulmányozását írják elő. Ezeket itt találjátok: 
\startalignment[middle]
\url[segedanyag]
\stopalignment
Arra kérlek titeket, hogy ha igényt tartotok a segédanyagok elolvasásáért járó pontokra, akkor valóban olvassátok el és próbáljátok értelmezni is azokat.

\startsection[title=Összeadás és kivonás]

A harmadik alkalommal kiderült, hogy a pozitív és a negatív számokkal való vegyesen végzett összeadás és a kivonás nem megy jól.
Elmondtam, hogy ezt muszáj orvosolni, azt hiszem ebben mindnyájan egyetértettünk.
Számomra úgy tűnik, két oldalról van itt hiba.
Először is, sokszor fejben próbáljátok elvégezni a műveleteket amiket jobb lenne papíron, mert a fejszámolás még nem megy olyan biztosan.
Sokan itt-ott néhány tizessel vagy százassal elszámoljátok magatokat és meg vagytok győződve róla, hogy a kapott eredmény jó.
Keveseknél láttam szépen papíron levezetett számításokat.
A jövőben aki fejben számol és hibásan, az ne lepődjön meg, ha kap egy papíron elvégzendő számolási feladatot.


\startfeladat
Nyomtasd ki, olvasd el és foglald össze legalább 30 szóval az {\em Összeadás és kivonás} című segédanyagot.\feladatpont{10 pont}

Volt valami, amit nem értettél belőle? Ha igen, írd le legalább 20 szóval.\feladatpont{5 pont}
\stopfeladat

\startfeladat
Végezd el papíron az alábbi összeadásokat és kivonásokat:\feladatpont{2 pont/db}

{\starttabulate[|Mcw(\dimexpr0.5\textwidth-0.5em\relax)|Mcw(\dimexpr0.5\textwidth-0.5em\relax)|]
 \NC 71202+58336 \NC 687155+55940\NR 
 \NC 19689+37843 \NC 5324+799271\NR 
 \NC 51729+47023 \NC 3999+4018\NR 
 \NC 7537+37751 \NC 152294+786947\NR 
 \NC 23718+62296 \NC 5065+26544\NR 
 \NC 63182-62319 \NC 85114-84029\NR 
 \NC 69147-9956 \NC 91047-55782\NR 
 \NC 621198-4320 \NC 784651-75774\NR 
 \NC 27820-13681 \NC 30904-8089\NR 
 \NC 483422-97863 \NC 74972-66059\NR 
\stoptabulate
}  
\stopfeladat

\startfeladat
Számold ki papíron vagy fejben az alábbi kifejezéseket:\feladatpont{2 pont/db}

{\starttabulate[|Mcw(\dimexpr0.5\textwidth-0.5em\relax)|Mcw(\dimexpr0.5\textwidth-0.5em\relax)|]
 \NC 39-46+0 \NC (-51)+(-55)-(-548)\NR 
 \NC 962+(-82)+(-36) \NC (-38)+(-83)-(-79)\NR 
 \NC (-26)-96-444 \NC (-35)+(-893)-(-2)\NR 
 \NC (-76)-86+(-636) \NC 0+(-2)-0\NR 
 \NC 39+(-9)+0 \NC 711-(-696)+(-1)\NR 
\stoptabulate
}  
\stopfeladat

\startfeladat
Számítsd ki 1-től 100-ig a természetes számok összegét.\feladatpont{20 pont}
\stopfeladat

\startfeladat
Sorszámozd a magyar ábécé betűit ($\text{A}=1; \text{Á}=2;\dots$), majd számold ki a keresztneved összegét. \feladatpont{10 pont}
\stopfeladat

\startsection[title=Prímszámok, prímtényezőkre bontás]

Beszéltünk a prímszámokról és az összetett számok prímtényezőkre bontásáról, de úgy érzem még barátkoznotok kell ezzel a témával.
A prímszámok azért fontosak, mert ők a többi szám építőkövei, rájuk épül az oszthatóság, a törtek egyszerűsítése, több szám közös többszöröseinek megtalálása, de az adatok titkosítása is. Nem árt, ha ismeri őket az ember.

\startfeladat
Nyomtasd ki, olvasd el és foglald össze legalább 30 szóval a {\em Prímszámok} című segédanyag {\em Mik azok a prímszámok?} és {\em Prímtényezőkre bontás} című fejezeteit.\feladatpont{10 pont}

Volt valami, amit nem értettél belőle? Ha igen, írd le legalább 20 szóval.\feladatpont{5 pont}

Olvasd el és foglald össze legalább 30 szóval ugyanennek a segédanyagnak az {\em Eratosztenész szitája} és a {\em Mennyi prímszám van?} című fejezeteit.\feladatpont{10 pont}

Volt valami, amit nem értettél belőle? Ha igen, írd le legalább 20 szóval.\feladatpont{5 pont}
\stopfeladat

\startfeladat
Rostáld ki a prímszámokat legalább 200-ig Erasztotenész szitájával. \feladatpont{20 pont}
\stopfeladat

\startfeladat
Végezd el az alábbi összetett számok prímtényezős felbontását és írd le őket prímszámtényezőkre bontva:\feladatpont{2 pont/db}

{\starttabulate[|Mcw(\dimexpr0.25\textwidth-0.75em\relax)|Mcw(\dimexpr0.25\textwidth-0.75em\relax)|Mcw(\dimexpr0.25\textwidth-0.75em\relax)|Mcw(\dimexpr0.25\textwidth-0.75em\relax)|]
 \NC 912 \NC 560 \NC 1176 \NC 2784\NR 
 \NC 168 \NC 1824 \NC 34680 \NC 2600\NR 
 \NC 416 \NC 4914 \NC 24 \NC 12\NR 
 \NC 240 \NC 792 \NC 2130 \NC 748\NR 
 \NC 40964 \NC 32 \NC 990 \NC 336\NR 
\stoptabulate
}  
\stopfeladat

\emptylines[1]\noindentation Hosszúhetény, 2018. május 5.

\stopsubject

\stopbodymatter
\stopproduct
\stopproject
