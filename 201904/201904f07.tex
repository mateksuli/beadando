\startproject iras
\mainlanguage[hu]
\setupalign[
  justified,
  nothanging,
  nohz,
  hyphenated,
  morehyphenated,
  tolerant,
]
\setupinterlinespace[height=0.75,depth=0.25]
\setuplayout[
  grid=no,
  location=middle,
]
\setupformulas[align=middle]
%\setupmathalignment[grid=no]

\def\PagenumberingCommand#1{\doifnot\pagenumber1{#1}}
\setuppagenumbering[
  location={footer,middle},
  command=\PagenumberingCommand,
]
\setuppapersize[A4]
\setuplayout[
    backspace=30mm,
    width=150mm,
    topspace=30mm,
    header=0mm,
    footer=15mm,
    footerdistance=0mm,
    bottom=0mm,
    bottomdistance=0mm,
    height=247mm
]

% Betűkészlet
%\setupbodyfont[libertinus,12pt]
\setupbodyfont[12pt]

% Vékony spácium bizonyos karakterek előtt (:;?!)
\definecharacterspacing [magyarpunctuation]
\setupcharacterspacing [magyarpunctuation] ["0021] [left=.1,alternative=1] % ! % strip preceding space(char)
\setupcharacterspacing [magyarpunctuation] ["003A] [left=.1,alternative=1] % : % strip preceding space(char)
\setupcharacterspacing [magyarpunctuation] ["003B] [left=.1,alternative=1] % ; % strip preceding space(char)
\setupcharacterspacing [magyarpunctuation] ["003F] [left=.1,alternative=1] % ? % strip preceding space(char)

% A magyar nyelv beállításai
\startsetups[magyar]
  % Vékony spácium bizonyos karakterek előtt (:;?!)
  \setcharacterspacing[magyarpunctuation]
  \setupindenting[%
    yes,% A bekezdéseket behúzással kezdjük.
    %next,% Az első bekezdés nincs behúzva.
    2em% Közepes méretű (átmeneti megoldás: igazából a mérete 24 cicerós sorig 1 kvirt, nagyobbbnál 2 kvirt kellene legyen -> TENNIALÓ)
  ]
\stopsetups

\setuplanguage[hu][%
  setups=magyar,% Érvényesíti a fent megadott beállításokat.
  spacing=packed% Frenchspacing (Gyurgyák 319. o.: egyenletes szóközök).
                % http://wiki.contextgarden.net/French_spacing).
]

% TENNIVALÓ: csak magyar nyelvre
% Idézetek (Gyurgyák, 86--87. o.).
\definedelimitedtext[quote][location=text]
\setupdelimitedtext[quote:1][
  left={\lowerleftdoubleninequote},
  right={\upperrightdoubleninequote},
  spaceafter=0
]
\setupdelimitedtext[quote:2][
  left={\rightguillemot\nobreak\hskip-.07em},
  right={\kern-0.03em\leftguillemot},
  spaceafter=0
]
\setupdelimitedtext[quote:3][
  left={\upperleftsingleninequote},
  right={\upperrightsingleninequote},
  spaceafter=0
]

\definebodyfontenvironment[default][em=italic]

\defineframedtext[kerdes][align=center,offset=0.5ex,style=italic,width=\dimexpr0.8\dimexpr\makeupwidth]

\defineframedtext[kivonat][offset=0.5ex, frame=off,style=italic,width=\dimexpr0.8\dimexpr\makeupwidth]

\definehead[cim][chapter]
\setuphead[cim][number=no,align=middle,after={},]

\define[2]\sectioncommand{\hbox{#1. #2}}
\setuphead[section][
  sectionsegments=section,
  command=\sectioncommand,
]

\define[2]\feladatcommand{\hbox{Feladat #1 \enskip #2}}
\definehead[feladat][subsection]
\setuphead[feladat][
    number=yes,
    textdistance=0pt,
    alternative=text,
    style=bf,
    commandafter={\enskip},
    command=\feladatcommand,
    sectionsegments=3:100,
    indentnext=no,
    beforesection={\setupindenting[no]},
    aftersection={\setupindenting[yes]},
]

\definebar[mathubar][underbar]  % enélkül esetleg elcsúszik az aláhúzás

%\showgrid[all]



% https://tex.stackexchange.com/a/459615/50554
\def\feladatpont#1{{\unskip\nobreak\hfil\penalty50
  \hskip2em\hbox{}\nobreak\hfil #1\/ 
  \parfillskip=0pt \finalhyphendemerits=0 \par}}

\defineitemgroup[alfeladat]
\setupitemgroup[alfeladat][each][right=),margin=1em,before={},inbetween={},after={},stopper=,style=italic]

\startproduct iras

\useURL[segedanyag][https://github.com/matektankor/segedanyag]

\startbodymatter

\cim{Beadandó dolgozat feladatok}
\startlinealignment[middle]
ötödik osztály, 2019. április, {7/12}
\stoplinealignment

\blank[2*big]

\noindentation Kedves tanulók! 
\blank
\noindentation Megpróbáltam nektek elmesélni, hogy a matematika, ami kizárólag a logikus emberi gondolkodás gyermeke, mennyire erős és örök.

Valóban, a matematikát nem lehet megúszni gondolkodás nélkül.
Azok a problémák, amivel a matematika szembesít titeket, gyakran első ránézésre teljesen megoldhatatlannak tűnnek.
Meg kell szokni, hogy az ember valahogy elinduljon a remélt megoldás felé, annak ellenére, hogy még nem látja a végét.
Meg kell szokni, hogy törje az ember a fejét egy probléma megoldásán.
Talán nem is a matematikai ismeret a legtöbb, amit a matematikától kaphattok, hanem a problémamegoldás képessége és a logikus gondolkodással való megbarátkozás.

Sokakat ez riaszt, és jobbnak látják, ha azt mondják maguknak, hogy ők ezeket a típusú problémákat nem akarják megoldani, őket ez nem érdekli.
George Bernard Shaw mondta: \language[en] ,,Two percent of the people think; three percent of the people think they think; and ninety-five percent of the people would rather die than think.'' \language[hu]

Én arra buzdítalak titeket, hogy eddzétek az agyatokat, nehogy eltunyuljon.
A matematika ebben segít.
Mint minden edzés, először kellemetlen ez is.
De ha nem adjátok fel, hamarosan beérik és meglepően szép gyümölcsöket fog teremni.
E~gyümölcsök leszakításáért persze azután is meg kell majd dolgozni, de az már olyan lesz, mint egy kellemes tavaszi szüret.

Sokat gondolkodtam azon, hogy milyen formában legyen a számonkérés. 
Úgy érzem, hogy voltak közös óráink amik különösen kellemesen teltek, és olyanok is, amiket jobb elfelejteni mindkét félnek.
A matematika nagyon kevés alaptörvényre épített hatalmas logikai vár, amit élvezettel alkottak olyan emberek akiket ez vonz.
A tanár azon dolgozik, hogy úgy mutassa be a tantárgyát, hogy vonzalmat ébresszen, hiszen amit nem ismerünk azt nem is szerethetjük.
Azok voltak a kellemes óráink, amik játékosan teltek és amelyeken inkább a matematika megkedveltetése volt a célom.
Többször előfordult, hogy egy ilyen jól eltöltött óra útán tévesen úgy kezeltelek titeket, mintha már egy igazán érdeklődő csapatom volna, és ebben a felfogásban próbáltam valami mélyebb és absztraktabb témát bemutatni.
Na ezek az órák sikerültek kellemetlenre.
Igyekszem ezt a hullámzást megszűntetni és arra alapozni, ami működik: a matematika játékos oldalára, illetve arra, hogy senkitől ne kívánjak erején felüli teljesítményt.

Ebben a szellemben írtam újra ezt a feladatsort az utolsó napokban.
Felejtsétek el, amit előzetesen mondtam a beadandó dolgozatról, mert változtattam a számonkérés szabályain.

Először is, {\em a felkínált feladatok közül szabadon válogathattok}.
Mindenki eldöntheti, hogy melyik feladatokat dolgozza ki.
A lényeg, hogy 100 pont kell az ötös, 75 a négyes, 50 a hármas, és 25 pont a kettes osztályzathoz.
Minthogy a hibás megoldások csak töredékpontokkal járnak, így aki biztosra akar menni a jegyét illetően, az nyugodtan megcélozhat 100 pontnál többet is.

Néhány szabály azonban van!
A dolgozatot külön lapon vagy lapokon kérem beadni.
Az egyes feladatok kidolgozásai és megoldásai jól azonosíthatók legyenek.
Minden esetben tollal dolgozzatok!
A munkalapokon ki ne satírozzatok semmit, egy vagy két vonallal áthúzni viszont természetesen bármit szabad.
Nem baj, ha akár oldalakon keresztül is nekiestek egy-egy feladat megoldásának, százszor is kihúzva a sikertelen próbálkozásokat; minden ilyen lapot adjatok be, mert a kitartó törekvést értékelni fogom, higyjétek el.
Ha viszont egy összevisszaság közepén rábukkantok a helyes megoldásra, azt szépen, elhatárolt területen emeljétek ki, akár annak az árán is, hogy még egyszer letisztult formában le kell írjátok az egész levezetést.
Nem kell viszont minden eredményt átvezetni egy külön lapra.

Vannak feladatok, melyek bizonyos segédanyagok tanulmányozását írják elő. Ezeket itt találjátok:

\framed[frame=off,align=middle,width=\textwidth,offset=4pt]{
\framed[location=middle,offset=4pt]{\url[segedanyag]}
}

Arra kérlek titeket, hogy ha igényt tartotok a segédanyagok elolvasásáért járó pontokra, akkor valóban olvassátok el és próbáljátok értelmezni is azokat.

A dolgozatra egy jegyet fogtok kapni és a beadási határidő 2019{.} április 30.







\startsection[title=Összeadás és kivonás]

A harmadik alkalommal kiderült, hogy a pozitív és a negatív számokkal való vegyesen végzett összeadás és kivonás nem megy jól.
Elmondtam, hogy ezt muszáj orvosolni, azt hiszem ebben mindnyájan egyetértettünk.
Számomra úgy tűnik, két oldalról van itt hiba.
Először is, sokszor fejben próbáljátok elvégezni a műveleteket amiket jobb lenne papíron, mert a fejszámolás még nem megy olyan biztosan.
Sokan itt-ott néhány tizessel vagy százassal elszámoljátok magatokat és meg vagytok győződve róla, hogy a kapott eredmény jó.
Keveseknél láttam szépen papíron levezetett számításokat.
A jövőben aki fejben számol és hibásan, az ne lepődjön meg, ha kap egy papíron elvégzendő számolási feladatot.


\startfeladat
Nyomtasd ki, olvasd el, és foglald össze legalább 30 szóval az {\em Összeadás és kivonás} című segédanyagot.\feladatpont{10 pont}

Volt valami, amit nem értettél belőle? Ha igen, írd le legalább 20 szóval.\feladatpont{5 pont}
\stopfeladat

\startfeladat
Két szám összege 847. Ha az egyik szám végéről elhagyjuk a nullát, akkor a másik számot kapjuk. Melyik ez a két szám?
\feladatpont{10 pont}
\stopfeladat

\startfeladat
Végezd el {\em papíron} az alábbi összeadásokat és kivonásokat:\feladatpont{2 pont/db}

{\starttabulate[|Mcw(\dimexpr0.5\textwidth-0.5em\relax)|Mcw(\dimexpr0.5\textwidth-0.5em\relax)|]
 \NC 9993+75791 \NC 973854+943945\NR 
 \NC 30259+2717 \NC 180513+1748\NR 
 \NC 3630+55206 \NC 4566+900205\NR 
 \NC 7099+32374 \NC 92797+80887\NR 
 \NC 13738+56527 \NC 2245+610578\NR 
 \NC 337020-33523 \NC 6059-4789\NR 
 \NC 58176-9523 \NC 755894-72182\NR 
 \NC 723092-6477 \NC 773166-729825\NR 
 \NC 285370-216769 \NC 58236-50801\NR 
 \NC 39076-4654 \NC 26022-1325\NR 
\stoptabulate
}  
\stopfeladat

\startfeladat
Számold ki papíron vagy fejben az alábbi kifejezéseket:\feladatpont{2 pont/db}

{\starttabulate[|Mcw(\dimexpr0.5\textwidth-0.5em\relax)|Mcw(\dimexpr0.5\textwidth-0.5em\relax)|]
 \NC 67+(-8)+4 \NC (-52)-50-0\NR 
 \NC 0+387+(-24) \NC 494-(-87)+(-43)\NR 
 \NC (-77)-1+(-4) \NC (-7)-(-56)+15\NR 
 \NC (-92)+(-27)+(-90) \NC (-44)-(-6)+(-37)\NR 
 \NC 632+(-52)+5 \NC (-5)+12+(-677)\NR 
\stoptabulate
}  
\stopfeladat

\startfeladat
Az alábbi összeadásban a betűk számjegyeket jelentenek: ugyanaz a betű ugyanazt a számjegyet, különböző betűk különböző számjegyeket.
Írd fel az összeadást számjegyekkel is!
\feladatpont{10 pont}
\startformula
{\framed[frame=off,align={flushright,nothyphenated,verytolerant},before={},after={}]{
$\text{\bf BDCE}$\\
\underbar{$+\text{\bf BDAE}$}\\
$\text{\bf AECBE}$\\
}}
\stopformula
\stopfeladat

\startfeladat
Számítsd ki 1-től 100-ig a természetes számok összegét.\feladatpont{20 pont}
\stopfeladat

\startfeladat
Sorszámozd a kiterjesztett magyar ábécé betűit ($\text{A}=1; \text{Á}=2;\dots$), majd számold ki a keresztneved összegét. \feladatpont{10 pont}
\stopfeladat








\startsection[title=Prímszámok, prímtényezőkre bontás]

Beszéltünk a prímszámokról és az összetett számok prímtényezőkre bontásáról, de úgy érzem még barátkoznotok kell ezzel a témával.
A prímszámok azért fontosak, mert ők a többi szám építőkövei, rájuk épül az oszthatóság, a törtek egyszerűsítése, több szám közös többszöröseinek megtalálása, de az adatok titkosítása is. Nem árt, ha ismeri őket az ember.

\startfeladat
Nyomtasd ki, olvasd el, és foglald össze legalább 30 szóval a {\em Prímszámok} című segédanyag {\em Mik azok a prímszámok?} és {\em Prímtényezőkre bontás} című fejezeteit.
\feladatpont{10 pont}

Volt valami, amit nem értettél belőle? Ha igen, írd le legalább 20 szóval.
\feladatpont{5 pont}

Olvasd el és foglald össze legalább 30 szóval ugyanennek a segédanyagnak az {\em Eratosztenész szitája} és a {\em Mennyi prímszám van?} című fejezeteit.
\feladatpont{10 pont}

Volt valami, amit nem értettél belőle? Ha igen, írd le legalább 20 szóval.
\feladatpont{5 pont}
\stopfeladat

\startfeladat

Marci három dobókockával játszott.
Egyik dobása után örömmel mondta nővérének, Sárinak:

-- Képzeld, sikerült úgy dobnom, hogy mindhárom kockán a felül lévő pöttyök száma prímszám és a három szám összege is prímszám, méghozzá 10-nél nagyobb.

-- Akkor biztosan van köztük kettő, amelyiken azonos számú pötty van fölül -- válaszolta Sári.

Igaza volt-e Sárinak és hány pötty lehetett a kockák felső lapján, ha Marci állítása igaz volt?
\feladatpont{8 pont}
\stopfeladat

\startfeladat
Rostáld ki a prímszámokat legalább 200-ig Erasztotenész szitájával.
\feladatpont{20 pont}
\stopfeladat

\startfeladat
A {\bf MALOM} szó egy ötjegyű számot helyettesít.
Azonos betűk azonos számokat, különböző betűk különböző számokat jelentenek.
A betűknek megfelelő számok mindegyike prímszám, azonkívül az öt szám összege is.
Törzsszám továbbá a {\bf MA} betűknek megfelelő kétjegyű szám és az {\bf MLO} betűknek megfelelő háromjegyű szám is.
Melyik lehet ez az ötjegyű szám?
\feladatpont{12 pont}
\stopfeladat

\startfeladat
Végezd el az alábbi összetett számok prímtényezős felbontását és írd le őket prímszámtényezőkre bontva:
\feladatpont{2 pont/db}

{\starttabulate[|Mcw(\dimexpr0.25\textwidth-0.75em\relax)|Mcw(\dimexpr0.25\textwidth-0.75em\relax)|Mcw(\dimexpr0.25\textwidth-0.75em\relax)|Mcw(\dimexpr0.25\textwidth-0.75em\relax)|]
 \NC 8 \NC 112 \NC 1400 \NC 10296\NR 
 \NC 63 \NC 912 \NC 1140 \NC 3040\NR 
 \NC 94710 \NC 408 \NC 1352 \NC 4784\NR 
 \NC 140 \NC 20 \NC 60 \NC 32200\NR 
 \NC 966 \NC 735 \NC 6375 \NC 80\NR 
\stoptabulate
}  
\stopfeladat

\startfeladat
Adj meg tíz olyan egymást követő természetes számot, amelyek között
\startalfeladat[a]
    \item nincs egyetlen törzsszám sem;\feladatpont{3 pont}
    \item pontosan egy törzsszám van;\feladatpont{3 pont}
    \item pontosan kettő törzsszám van;\feladatpont{3 pont}
    \item pontosan három törzsszám van;\feladatpont{3 pont}
    \item pontosan négy törzsszám van.\feladatpont{3 pont}
\stopalfeladat
Legfeljebb hány törzsszám lehet 10 egymást követő természetes szám között?
\feladatpont{3 pont}
\stopfeladat


\emptylines[1]\noindentation Hosszúhetény, 2019{.} április 14.

\stopsubject

\stopbodymatter
\stopproduct
\stopproject
