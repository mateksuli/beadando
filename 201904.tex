\startproject iras
\mainlanguage[hu]
\setupalign[
  justified,
  nothanging,
  nohz,
  hyphenated,
  morehyphenated,
  tolerant,
]
\setupinterlinespace[height=0.75,depth=0.25]
\setuplayout[
  grid=no,
  location=middle,
]
\setupformulas[align=middle]
%\setupmathalignment[grid=no]

\def\PagenumberingCommand#1{\doifnot\pagenumber1{#1}}
\setuppagenumbering[
  location={footer,middle},
  command=\PagenumberingCommand,
]
\setuppapersize[A4]
\setuplayout[
    backspace=30mm,
    width=150mm,
    topspace=30mm,
    header=0mm,
    footer=15mm,
    footerdistance=0mm,
    bottom=0mm,
    bottomdistance=0mm,
    height=247mm
]

% Betűkészlet
%\setupbodyfont[libertinus,12pt]
\setupbodyfont[12pt]

% Vékony spácium bizonyos karakterek előtt (:;?!)
\definecharacterspacing [magyarpunctuation]
\setupcharacterspacing [magyarpunctuation] ["0021] [left=.1,alternative=1] % ! % strip preceding space(char)
\setupcharacterspacing [magyarpunctuation] ["003A] [left=.1,alternative=1] % : % strip preceding space(char)
\setupcharacterspacing [magyarpunctuation] ["003B] [left=.1,alternative=1] % ; % strip preceding space(char)
\setupcharacterspacing [magyarpunctuation] ["003F] [left=.1,alternative=1] % ? % strip preceding space(char)

% A magyar nyelv beállításai
\startsetups[magyar]
  % Vékony spácium bizonyos karakterek előtt (:;?!)
  \setcharacterspacing[magyarpunctuation]
  \setupindenting[%
    yes,% A bekezdéseket behúzással kezdjük.
    %next,% Az első bekezdés nincs behúzva.
    medium% Közepes méretű (átmeneti megoldás: igazából a mérete 24 cicerós sorig 1 kvirt, nagyobbbnál 2 kvirt kellene legyen -> TENNIALÓ)
  ]
\stopsetups

\setuplanguage[hu][%
  setups=magyar,% Érvényesíti a fent megadott beállításokat.
  spacing=packed% Frenchspacing (Gyurgyák 319. o.: egyenletes szóközök).
                % http://wiki.contextgarden.net/French_spacing).
]

% TENNIVALÓ: csak magyar nyelvre
% Idézetek (Gyurgyák, 86--87. o.).
\definedelimitedtext[quote][location=text]
\setupdelimitedtext[quote:1][
  left={\lowerleftdoubleninequote},
  right={\upperrightdoubleninequote},
  spaceafter=0
]
\setupdelimitedtext[quote:2][
  left={\rightguillemot\nobreak\hskip-.07em},
  right={\kern-0.03em\leftguillemot},
  spaceafter=0
]
\setupdelimitedtext[quote:3][
  left={\upperleftsingleninequote},
  right={\upperrightsingleninequote},
  spaceafter=0
]

\definebodyfontenvironment[default][em=italic]

\defineframedtext[kerdes][align=center,offset=0.5ex,style=italic,width=\dimexpr0.8\dimexpr\makeupwidth]

\defineframedtext[kivonat][offset=0.5ex, frame=off,style=italic,width=\dimexpr0.8\dimexpr\makeupwidth]

\definehead[feladat][subsubsubject]
\setuphead[feladat][
    number=no,
    textdistance=0pt,
    alternative=text,
    style=bf,
    commandafter={\hbox{.\quad}},
]

\definebar[mathubar][underbar]  % enélkül esetleg elcsúszik az aláhúzás

%\showgrid[all]

\definehead[cim][chapter]
\setuphead[cim][number=no,align=middle,after={},]

\startproduct iras
\startbodymatter

\cim{Elolvasandó és beadandó}
\startlinealignment[middle]
2019. április\\
{{azonosító}}
\stoplinealignment

\blank[2*big]

\noindentation Kedves tanulók! 
\blank
\noindentation Megpróbáltam nektek elmesélni, hogy a matematika, ami kizárólag a logikus emberi gondolkodás gyermeke, mennyire erős és örök.
Valóban, a matematikát nem lehet gondolkodás nélkül megúszni, csak ha az ember megszokja, hogy itt egy probléma és törnöm kell a fejem rajta, ha meg akarom oldani.
Sokakat ez riaszt, és jobbnak látják, ha azt mondják maguknak, hogy ők ezeket a típusú problémákat nem akarják megoldani, őket ez nem érdekli.
George Bernard Shaw mondta: \language[en] ,,Two percent of the people think; three percent of the people think they think; and ninety-five percent of the people would rather die than think.'' \language[hu]
Én arra buzdítalak titeket, hogy eddzétek az agyatokat, nehogy eltunyuljon.
A matematika ebben segít.
Mint minden edzés, először kellemetlen ez is.
De ha nem adjátok fel, hamarosan beérik és meglepően szép gyümölcsöket fog teremni.
E gyümölcsök leszakításáért persze aztán is meg kell majd dolgozni, de az már olyan lesz, mint egy kellemes tavaszi szüret.

A dolgozatot külön lapon vagy lapokon kérem beadni.
Az egyed feladatok kidolgozásai és megoldásai jól azonosíthatók legyenek.
Minden esetben tollal dolgozzatok!
A munkalapokon ki ne satírozzatok semmit, egy vagy két vonallal áthúzni viszont természetesen bármit szabad.
Nem baj, ha akár oldalakon keresztül is nekiestek egy-egy feladat megoldásának, százszor is kihúzva a sikertelen próbálkozásokat; minden ilyen lapot adjatok be, a kitartó törekvést értékelni fogom, higyjétek el.
Ha viszont egy ilyen összevisszaság közepén rábukkantok a helyes megoldásra, azt szép formában, elhatárolt területen emeljétek ki, akár annak az árán is, hogy még egyszer letisztult formában le kell írjátok az egész levezetést.

Az értékelést egymáshoz képest kapjátok majd. A két legjobb eredmény 5-öst, a két legrosszabb 2-est fog érni. Az átlagon felüliek 4-est, az átlag alattiak 3-ast.

\startsubject[title=Összeadás és kivonás]

A harmadik alkalommal kiderült, hogy a pozitív és a negatív számokkal való vegyesen végzett összeadás és a kivonás nem megy jól.
Elmondtam, hogy ezt muszáj orvosolni, azt hiszem ebben mindnyájan egyetértettünk.
Számomra úgy tűnik, két oldalról van itt hiba.
Először is, sokszor fejben próbáljátok elvégezni a műveleteket amiket jobb lenne papíron, mert a fejszámolás még nem megy olyan biztosan.
Sokan itt-ott néhány tizessel vagy százassal elszámoljátok magatokat és meg vagytok győződve róla, hogy a kapott eredmény jó.
Keveseknél láttam szépen papíron levezetett számításokat.
A jövőben aki fejben számol és hibásan, az fog kapni egy papíron elvégzendő számolási feladatot.
Most azonban mindenki kap néhányat, de először álljon itt emlékeztetőül, hogy mi az összeadás és a kivonás és miképpen kell összeadni és kivonni:

Két munkás közül az egyik egy óra alatt 34 szegecset illesztett be, a másik 27-et. Hány szegecset illesztettek be együtt egy óra alatt?

Annyi szegecset illesztettek be, mint amennyi 34 és 27 együttvéve, amelynek a matematikai írásmódja: $34+27=61$ (olv.: harmincnégy {\em plusz} huszonhét egyenlő hatvaneggyel). Ez a művelet az összeadás, műveleti jele: $+$. A 34-et és a 27-et összeadandóknak vagy {\em tagoknak}, az eredményt, a 61-et pedig {\em összegnek} nevezzük.

\startformula \startmathalignment[n=1]
\NC \text{összeadandó} + \text{összeadandó} = \text{összeg} \NR[+]
\NC \text{tag} + \text{tag} = \text{összeg} \NR[+]
\stopmathalignment \stopformula

Az összeadás első fontos tulajdonsága, hogy a {\em tagok sorrendje felcserélhető}, az összegük nem változik, azaz az összeg független az összeadás sorrendjétől. Ez az összeadás felcserélési (kommutatív) törvénye.

\startformula 3+5=5+3=8 \stopformula

Az összeadás második fontos tulajdonsága, hogy a tagokat tetszés szerint csoportosíthatjuk, az összeg nem változik. Például három tag esetén bármelyik kettő összegéhez hozzáadhatjuk a harmadik tagot, az összeg nem változik. Ez az összeadás csoportosítási (asszociatív) törvénye.

\startformula (3+5)+7=3+(5+7)=15 \stopformula

Az összeadást többjegyű összeadandók esetén a következő módon végezzük: a számokat úgy írjuk egymás alá, hogy az azonos helyi értékű számjegyek egymás alá kerüljenek.
Az utolsó összeadandót aláhúzzuk. Például:

\startformula
%{\framed[width=\textwidth,frame=off,align=middle,after={\blank}]{
{\framed[frame=off,align={flushright,nothyphenated,verytolerant},before={},after={}]{
$2435$\\
$247$\\
\underbar{$+1413$}\\
$4095$\\
}}
%}}
\stopformula

Az összeadást a legkisebb helyi értékű jegyenél kezdjük: $3+7+5=15$.
Az eredmény $5$ egyesét leírjuk az egyesek alá, $1$ tizesét pedig a tízesekhez adjuk hozzá: $1+1+4+3=9$.
A százasok összege: $4+2+4=10$, tehát a százasok oszlopa alá $0$-t írunk, és az $1$ ezrest az ezresekhez adjuk hozzá: $1+1+2=4$.
A $4$-et az ezresek alá írjuk. 
A három adott többjegyű szám összege tehát 4095.
Az összeadást felülről lefelé is végezzük el, mert számításunk helyességét így ellenőrizhetjük.

15\,m vörösréz huzalból elhasználtunk 9\,m-t. Hány méter maradt?

Annyi maradt, amennyivel több a 15\,m, mint a 9\,m, ennek matematikai írásmódja: $15-9=6$ (olv.: tizenöt {\em mínusz} kilenc egyenlő hattal).
Ez a művelet a kivonás, műveleti jele: $-$.

Azt a számot kerestük, amelyhet 9-hez adva összeül 15-öt kapunk.
A kivonás az összeadás fordított művelete. Kivonásnál ismert az egyik összeadandó és az összeg és keressük a másik összeadandót. Az összeget {\em kisebbítendőnek}, az ismert összeadandót {\em kivonandónak}, az ismeretlen összeadandót pedig {\em különbségnek} vagy {\em maradéknak} nevezzük.

\startformula \startmathalignment[n=1]
\NC \text{kisebbítendő} - \text{kivonandó} = \text{különbség} \NR[+]
\NC \text{kisebbítendő} - \text{kivonandó} = \text{maradék} \NR[+]
\stopmathalignment \stopformula

A kivonás helyes elvégzésének próbája az összeadás:

\startformula \text{kivonandó} + \text{különbség} = \text{kisebbítendő} \stopformula

Fontos tulajdonsága a kivonásnak, hogy a különbség értéke nem változik, ha a kisebbítendőhöz és a kivonandóhoz {\em ugyanazt a számot hozzáadjuk, vagy mindkettőből ugyanazt kivonjuk}.

\startformula \startmathalignment[n=1]
\NC 13-11=2 \NR[+]
\NC (13+3)-(11+3)=16-14=2 \NR[+]
\NC (13-3)-(11-3)=10-8=2 \NR[+]
\stopmathalignment \stopformula

A kisebbítendő növelésével ill{.} csökkentésével a különbség ugyanannyival nő ill{.} csökken. A kivonandó növelésével a különbség ugyanannyival csökken, csökkentésével ugyanannyival nő.

Több számjegyű számok kivonását úgy végezzük el, hogy a számokat helyi értéküknek megfelelően egymás alá írjuk és a kivonást a legkisebb helyi értékű jegynél kezdjük:

\startformula
%{\framed[width=\textwidth,frame=off,align=middle,after={\blank}]{
{\framed[frame=off,align={flushright,nothyphenated,verytolerant},before={},after={}]{
$5832$\\
\underbar{$-3521$}\\
$2311$\\
}}
%}}
\stopformula

1 meg 1 az 2 (leírjuk az 1-et), 2 meg 1 az 3 (leírjuk az 1-et), 5 meg 3 az nyolc (leírjuk a 3-at), 3 meg 2 az öt (leírjuk az 2-t). A különbség: 2311.

Ellenőrzés: $2311+3521=5832$.

Amikor a kisebbítendő valamelyik számjegye kisebb, mint a kivonandó ugyanolyan helyi értékű számjegye, akkor felhasználjuk a már említett törvényt, amely szerint a különbség nem változik, ha a kisebbítendőt és a kivonandót ugyanazzal a számmal növeljük.

\startformula
%{\framed[width=\textwidth,frame=off,align=middle,after={\blank}]{
{\framed[frame=off,align={flushright,nothyphenated,verytolerant},before={},after={}]{
$3762$\\
\underbar{$-1835$}\\
$1927$\\
}}
%}}
\stopformula

Mivel 2 egyesből 5 egyest nem vonhatunk ki, a kisebbítendőhöz 10 egyest hozzáadunk, így 12 egyesünk lesz, és hogy a különbség ne változzék, a kivonandóhoz ugyancsak 10 egyest, de 1 tízes alakjában hozzáadunk a kivonandó tízeseihez is. Hasonlóan járunk el a százasoknál is, ahol 10 százassal (1 ezressel) növeljük a kisebbítendőt és 1 ezressel a kivonandót. A kivonást tehát így végezzük: 5 meg 7 az 12 (leírjuk a 7-et), 1 meg 3 az 4 meg 2 az 6 (leírjuk a 2-t), 8 meg 9 az 17 (leírjuk a 9-et), 1 meg 1 az 2, 2 meg 1 az 3 (leírjuk az egyet). A különbség vagy maradék: 1927. 

Próba: ha a maradékot a kivonandóhoz hozzáadva a kisebbítendőt kapjuk, helyes az eredményünk.

\startformula
%{\framed[width=\textwidth,frame=off,align=middle,after={\blank}]{
{\framed[frame=off,align={flushright,nothyphenated,verytolerant},before={},after={}]{
$1835$\\
\underbar{$+1927$}\\
$3762$\\
}}
%}}
\stopformula

Az $\mathbb{N}$ (természetes számok) számhalmazban a kivonást csak akkor végezhetjük el, ha a $\text{kisebbítendő}\geq\text{kivonandó}$ feltétel teljesül.

\startfeladat[title=Feladat I]
Végezd el az alábbi műveleteket:\hfill40 pont

{{feladatok1}}  
\stopfeladat

\startfeladat[title=Feladat II]
Számítsd ki 1-től 100-ig a természetes számok összegét. \hfill20 pont
\stopfeladat
\blank

Ez volt a kulimunka. Kellemetlen, de szükséges. Viszont észrevehettétek, hogy negatív számokkal most nem kellett bajlódnotok. Eljön az is majd, nem kell félni. Innentől érdekesebb feladatok jönnek.

\startfeladat[title=Feladat III]
Sorszámozd a magyar ábécé betűit ($\text{A}=1; \text{Á}=2;\dots$), majd számold ki a keresztneved összegét.\hfill10 pont
\stopfeladat
\blank

\startsubject[title=Prímszámok, prímtényezőkre bontás]

Beszéltünk a prímszámokról és az összetett számok prímtényezőkre bontásáról, de úgy érzem még barátkoznotok kell ezekkel.
A prímszámok azért fontosak, mert rájuk épül az oszthatóság, a törtek egyszerűsítése, több szám közös többszöröseinek megtalálása, de az adatok titkosítása is.
Lássuk!

A hinduk ősidőktől fogva kitűnő matematikusok, és sajátos képességeik vannak ezen a téren.
Mikor egyszer Hardy és Rámánudzsan Londonban taxin utazott, Hardy a taxi távozása után vette észre, hogy aktatáskáját a kocsiban felejtette.
Kéziratok lévén a táskában, ez kétségbe ejtette, de Rámánudzsan megnyugtatta, hogy a taxi száma 1729.
Hardy igen örült ennek, de nem hagyta nyugodni a kérdés, hogyan lehetett megjegyezni egy ilyen érdektelen számot.
Nem érdektelen ez a szám, felelte Rámánudzsan: ez a legkisebb olyan egész szám, amely kétféleképpen bontható fel két köbszám összegére, hiszen $10^3+9^3$ is és $12^3+1^3$ is 1729.

A hinduknak még a négyjegyű számok is ilyen külön sajátságokkal felruházott személyes ismerőseik.
Nálunk a kis számokat kezelik ilyen individuumok módjára: a 2-es nem a sok szürke szám egyike, hanem sok oldalról megismert különálló egyéniség: ő az első páros szám, $1+1$, 4-nek a fele stb. 
De akár 10-ig színezzük így a számokat, akár olyan messzeségekig, mint a hinduk, mindez csak szerény kis töredéke a végtelen számsornak, amely ezen túl szürkén hömpölyög tovább.



Tudjuk ugyan, hogy vannak páros számok, igen, minden második szám páros:
\startformula
1, \mathubar{2}, 3, \mathubar{4}, 5, \mathubar{6}, 7, \mathubar{8}, 9, \mathubar{10}, 11, \mathubar{12}, \dots
\stopformula

ugyanígy minden harmadik szám osztható 3-mal:
\startformula
1, 2, \mathubar{3}, 4, 5, \mathubar{6}, 7, 8, \mathubar{9}, 10, 11, \mathubar{12}, \dots
\stopformula

minden negyedik szám 4-gyel:
\startformula
1, 2, 3, \mathubar{4}, 5, 6, 7, \mathubar{8}, 9, 10, 11, \mathubar{12}, \dots
\stopformula

s. í. t., ezek azonban csak kisebb-nagyobb hullámokat jelentenek, melyek, ha egyszer megindultak, egyhangúan, egyformán gördülnek tovább.
Valóban nincs semmi váratlan, semmi egyéni szeszély, ami felélénkíthetné ezt az egyhangúságot?

De van: a prímszámok szeszélyes, szabályokba nem szorítható eloszlása. Emlékezzünk csak az oszthatóságra: 

\startformula \startmathalignment
\NC \text{10 összes osztói: } \NC 1, 2, 5, 10,\NR[+]
\NC \text{12 összes osztói: } \NC 1, 2, 3, 4, 6, 12, \NR[+]
\NC \text{közben } \text{11 összes osztói: } \NC 1, 11.\NR[+]
\stopmathalignment \stopformula

1-gyel és önmagával minden szám osztható; vannak számok, amelyek e kettőn kívül semmi mással: ilyen például a 11. Ezeket a számokat nevezik törzsszámoknak vagy prímszámoknak.

Az 1 e szempontból rendellenesen viselkedik; csak egy osztója van: 1, és ez egyszersmind önmaga. Ezért 1-et nem szokás a prímszámok közé sorolni. Az 1 neve {\em egység}.
Eszerint a legkisebb prímszám a 2, ez egyszersmind az egyetlen páros prímszám, mert minden páros szám osztható 2-vel és ez a szám prímszám voltát csak akkor nem rontja el, ha ez a 2-es osztó maga a szám.

Jelentőséget az ad a prímszámoknak, hogy minden más szám ezekből az építőkövekből rakható össze; éppen ezért nevezik a többi számot összetett számnak. Pontosabban úgy fogalmazható ez meg, hogy minden összetett szám csupa prímszám szorzataként állítható elő.

Próbáljuk például 60-at szorzatként írni fel:
\startformula
60 = 6 \cdot 10
\stopformula

Itt 6 és 10 is tovább bontható tényezőkre:

\startformula
6=2\cdot 3 \quad\text{és}\quad 10=2 \cdot 5,
\stopformula

ezeket beírva 6 és 10 helyére:

\startformula
60=2 \cdot 3 \cdot 2 \cdot 5
\stopformula

és itt már minden tényező prímsszám.

Másképp is hozzáfoghattunk volna ehhez, hiszen már láttuk, hogy 60-at nagyon sokféleképpen lehet két szám szorzataként felírni.
Ha ebből indulunk ki:

\startformula \startmathalignment
\NC \null \NC 60=4\cdot 15, \NR[+]
\NC \text{itt}\quad\NC 4=2\cdot 2 \quad\text{és}\quad 15=3 \cdot 5, \NR[+]
\NC \text{tehát}\quad \NC 60=2 \cdot 2 \cdot 3 \cdot 5, \NR[+]
\stopmathalignment \stopformula

ha pedig ezt a felbontást választjuk:

\startformula \startmathalignment[n=8,align={right,left,left,left,right,left,left,left}]
\NC \NC 60\NC=2\cdot 30, \NR[+]
\NC \text{akkor}\quad\NC 30\NC=5\cdot 6 \quad\NC\text{és itt}\quad \NC6\NC=2 \cdot 3, \quad\NC\text{tehát}\quad 30\NC=5 \cdot 2 \cdot 3, \NR[+]
\NC \text{vagy}\quad\NC 30\NC=2\cdot 15 \quad\NC\text{és itt}\quad \NC15\NC=3 \cdot 5, \quad\NC\text{tehát}\quad 30\NC=2 \cdot 3 \cdot 5, \NR[+]
\NC \text{vagy}\quad\NC 30\NC=3\cdot 10 \quad\NC\text{és itt}\quad \NC10\NC=2 \cdot 5, \quad\NC\text{tehát}\quad 30\NC=3 \cdot 2 \cdot 5; \NR[+]
\stopmathalignment \stopformula

látjuk tehát, hogy 30 mindenképpen a 2, 3 és 5 prímszámok szorzatára bomlik; e szorzatot írva 30 helyébe

\startformula
60=2 \cdot 2 \cdot 3 \cdot 5.
\stopformula

Bármi módon fogunk is hozzá, csak ugyanazon prímszámokra esik szét 60, legfeljebb más sorrendben lépnek fel ezek. Rendbe szedve és az egyenlő tényezők szorzatát hatványalakban írva

\startformula
60=2^{2} \cdot 3 \cdot 5.
\stopformula

Ugyanilyen könnyű bármely más összetett számot is felbontani ,,prímtényezőire’’ (és be lehet bizonyítani, hogy mindig csak egyféle felbontásra juthatunk).
Ha első pillanatra megakadunk azon, hogyan is fogjunk hozzá, gondoljuk meg, hogy a szám legkisebb osztója (1-en kívül) biztosan prímszám, mert ha az is összetett szám volna, akkor kellene önmagánál kisebb osztójának lenni, és ez persze az eredeti számban is megvolna maradék nélkül.
Tehát mindig a legkisebb osztót keresve, szépen legöngyölíthetők bármely szám prímtényezői, például:

\startformula \startmathalignment[n=2,align={right,left}]
\NC 90\NC=2 \,\, \cdot \,\, 45 \NR
\NC   \NC=2\cdot\overbrace{3\,\cdot\,15} \NR
\NC   \NC=2\cdot 3\cdot\overbrace{3\cdot 5} \NR
\stopmathalignment \stopformula

Egy ilyen felbontás jól bevilágít a szám szerkezetébe, pl. kiolvasható belőle, hogy 90 osztói 1-en kívül:
\startformula \startmathalignment[n=2,align={left,left}]
\NC \text{egytényezősök: } \NC 2, 3, 5, \NR
\NC \text{kéttényezősök: } \NC 2\cdot3=\mathubar{6},\quad 2\cdot5=\mathubar{10},\quad 3\cdot3=\mathubar{9},\quad 3\cdot5=\mathubar{15}, \NR
\NC \text{háromtényezősök: } \NC 2\cdot3\cdot3=\mathubar{18},\quad 2\cdot3\cdot5=\mathubar{30},\quad 3\cdot3\cdot5=\mathubar{45}, \NR
\NC \text{négytényezős: } \NC 2\cdot3\cdot3\cdot5=\mathubar{90}. \NR
\stopmathalignment \stopformula

Tehát érdemes a számok építőköveivel közelebbről is megismerkedni.
Próbáljuk felírni rendre a prímszámokat.
Már tudjuk, hogy a legkisebb prímszám a 2, és innen kezdve a páros számokat nyugodtan át lehet ugrani, hiszen ezek mind oszthatók 2-vel.
3 is, 5 is, 7 is prímszám; az ember szeretné rámondani, hogy a 9 is, de ez nem igaz, hiszen 9 osztható 3-mal.
Most azt gondolnók, hogy innen kezdve ritkulnak a prímszámok; ez megint nem igaz, mert 11 és 13 is prímszám.

Még a régi görögöktől maradt ránk egy szellemes ötlet, amely tévedés lehetősége nélkül, gépiesen állítja elő ezt a szeszélyes sorozatot: az ún{.} Eratosztenész-féle rosta.
Írjuk fel a számokat 2-től 50-ig; e sorozat első tagja látatlanban is biztosan prímszám, hiszen minden valódi osztója kisebb volna nála s így (1-en kívül) előtte szerepelne a sorozatban, előtte azonban nincs semmi. Most nézzük meg, hogy mi ez az első szám: 2. Minden második szám 2-nek többszöröse és így 2 kivételével nem prímszám, tehát innen kezdve húzzunk ki minden második számot:


% https://tex.stackexchange.com/a/129129/50554
\defineframed
    [bekarikazott]
    [
      corner=round,
      location=low,
      loffset=0.1em,
      roffset=0.1em,
      rulethickness=0.5mm,
    ]


\startuniqueMPgraphic{athuzott a}
pickup pencircle scaled 0.5mm ;
path p ; p := (0pt+3pt,0pt+3pt)--(\overlaywidth-3pt,\overlayheight-3pt) ;
draw p withcolor black ;
\stopuniqueMPgraphic
\defineoverlay[athuzott][\uniqueMPgraphic{athuzott a}]

\def\athuzott#1{\noindent\inframed[frame=off,background=athuzott,location=low]{#1} }

\startformula
\bekarikazott{2}, 3, \athuzott{4}, 5, \athuzott{6}, 7, \athuzott{8}, 9, \athuzott{10}, 11, \athuzott{12}
\stopformula

A legelső szám, ami 2 után épségben maradt, ismét csak prímszám lehet, hiszen csak előtte szereplő számnak lehetne többszöröse, előtte pedig csak olyan szám van, melynek többszöröseit kihúztuk.
Nézzük meg ezt a számot: 3.
Minden harmadik szám 3-nak többszöröse, tehát húzzunk ki innen kezdve minden harmadik számot (nem baj, hogy így egyes számokat kétszeresen is áthúzunk):

\startformula
\bekarikazott{2}, \bekarikazott{3}, \athuzott{4}, 5, \athuzott{6}, 7, \athuzott{8}, \athuzott{9}, \athuzott{10}, 11, \athuzott{12}
\stopformula

Ha 12-ig bezárólag vagyunk kíváncsiak a prímszámokra, akkor tovább már nem is kell mennünk, mert az első fennmaradó szám 5, és ennek a 3-szorosánál nagyobb többszörösei már túl vannak 12-n, kisebb többszörösei pedig már mind a kihúzott számok közt szerepelnek. 12-ig tehát a következő prímszámokat találtuk: $ 2, 3, 5, 7, 11$.

\startfeladat[title=Feladat IV]
Rostáld ki a prímszámokat legalább 200-ig. A lányok beadhatják a korábban tőlem kapott lapot, ami 4000-ig tartalmazza a számokat, ha azon eljutottak 200-ig.\hfill\hbox{40 pont}
\stopfeladat
\blank

Gépet is lehetne szerkeszteni, mely az itt adott utasítást végrehajtja, és így hibátlanul ontja a prímszámokat egy bizonyos határig.
Beszéltünk is róla, hogy ezzel a módszerrel keresik a prímszámokat a szuperszámítógépekkel.
Ez azonban mit sem változtat azon, hogy a prímszámok minden határon túl a legszeszélyesebben bukkannak fel újra meg újra.

Így például meg lehet mutatni, hogy bármilyen nagy réseket találhatunk köztük, ha elég messzire megyünk a számsorban. Például egy legalább 6 egységnyi rést, azaz hat egymást követő olyan számot, amelyek egyike sem prímszám, adnak a következő műveletek eredményei:


\emptylines[1]\noindentation Hosszúhetény, 2018. május 5.

\stopsubject

\stopbodymatter
\stopproduct
\stopproject
