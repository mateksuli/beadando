\startproject iras
\mainlanguage[hu]
\setupalign[
  justified,
  nothanging,
  nohz,
  hyphenated,
  morehyphenated,
  tolerant,
]
\setupinterlinespace[height=0.75,depth=0.25]
\setuplayout[
  grid=yes,
  location=middle,
]
\def\PagenumberingCommand#1{\doifnot\pagenumber1{#1}}
\setuppagenumbering[
  location={footer,middle},
  command=\PagenumberingCommand,
]
\setuppapersize[A4]
\setuplayout[
    backspace=30mm,
    width=150mm,
    topspace=30mm,
    header=0mm,
    footer=15mm,
    footerdistance=0mm,
    bottom=0mm,
    bottomdistance=0mm,
    height=247mm
]

% Betűkészlet
%\setupbodyfont[libertinus,12pt]
\setupbodyfont[12pt]

% Vékony spácium bizonyos karakterek előtt (:;?!)
\definecharacterspacing [magyarpunctuation]
\setupcharacterspacing [magyarpunctuation] ["0021] [left=.1,alternative=1] % ! % strip preceding space(char)
\setupcharacterspacing [magyarpunctuation] ["003A] [left=.1,alternative=1] % : % strip preceding space(char)
\setupcharacterspacing [magyarpunctuation] ["003B] [left=.1,alternative=1] % ; % strip preceding space(char)
\setupcharacterspacing [magyarpunctuation] ["003F] [left=.1,alternative=1] % ? % strip preceding space(char)

% A magyar nyelv beállításai
\startsetups[magyar]
  % Vékony spácium bizonyos karakterek előtt (:;?!)
  \setcharacterspacing[magyarpunctuation]
  \setupindenting[%
    yes,% A bekezdéseket behúzással kezdjük.
    %next,% Az első bekezdés nincs behúzva.
    medium% Közepes méretű (átmeneti megoldás: igazából a mérete 24 cicerós sorig 1 kvirt, nagyobbbnál 2 kvirt kellene legyen -> TENNIALÓ)
  ]
\stopsetups

\setuplanguage[hu][%
  setups=magyar,% Érvényesíti a fent megadott beállításokat.
  spacing=packed% Frenchspacing (Gyurgyák 319. o.: egyenletes szóközök).
                % http://wiki.contextgarden.net/French_spacing).
]

% TENNIVALÓ: csak magyar nyelvre
% Idézetek (Gyurgyák, 86--87. o.).
\definedelimitedtext[quote][location=text]
\setupdelimitedtext[quote:1][
  left={\lowerleftdoubleninequote},
  right={\upperrightdoubleninequote},
  spaceafter=0
]
\setupdelimitedtext[quote:2][
  left={\rightguillemot\nobreak\hskip-.07em},
  right={\kern-0.03em\leftguillemot},
  spaceafter=0
]
\setupdelimitedtext[quote:3][
  left={\upperleftsingleninequote},
  right={\upperrightsingleninequote},
  spaceafter=0
]

\defineframedtext[kerdes][align=center,offset=0.5ex,style=italic,width=\dimexpr0.8\dimexpr\makeupwidth]

\defineframedtext[kivonat][offset=0.5ex, frame=off,style=italic,width=\dimexpr0.8\dimexpr\makeupwidth]

%\showgrid[all]

\definehead[cim][chapter]
\setuphead[cim][number=no,align=middle,after={},]

\startproduct iras
\startbodymatter

\cim{Beadandó dolgozat}
\startlinealignment[middle]
2019. április
\stoplinealignment

\blank[2*big]

\noindentation Kedves tanulók! 
\blank
\noindentation Megpróbáltam nektek elmesélni, hogy a matematika, ami kizárólag a logikus emberi gondolkodás gyermeke, mennyire erős és örök.
Valóban, a matematikát nem lehet gondolkodás nélkül megúszni, csak ha az ember megszokja, hogy itt egy probléma és törnöm kell a fejem rajta, ha meg akarom oldani.
Sokakat ez riaszt, és jobbnak látják, ha azt mondják maguknak, hogy ők ezeket a típusú problémákat nem akarják megoldani, őket ez nem érdekli.
George Bernard Shaw mondta: \language[en] ,,Two percent of the people think; three percent of the people think they think; and ninety-five percent of the people would rather die than think.'' \language[hu]
Én arra buzdítalak titeket, hogy eddzétek az agyatokat, nehogy eltunyuljon.
A matematika ebben segít.
Mint minden edzés, először kellemetlen ez is.
De ha nem adjátok fel, hamarosan beérik és meglepően szép gyümölcsöket fog teremni.
E gyümölcsök leszakításáért persze aztán is meg kell majd dolgozni, de az már olyan lesz, mint egy kellemes tavaszi szüret.

\startsubject[title=Összeadás és kivonás]

A harmadik alkalommal kiderült, hogy a pozitív és a negatív számokkal való vegyesen végzett összeadás és a kivonás nem megy jól.
Elmondtam, hogy ezt muszáj orvosolni, azt hiszem ebben mindnyájan egyetértettünk.
Számomra úgy tűnik, két oldalról van itt hiba.
Először is, sokszor fejben próbáljátok elvégezni a műveleteket amiket jobb lenne papíron, mert a fejszámolás még nem megy olyan biztosan.
Sokan itt-ott néhány tizessel vagy százassal elszámoljátok magatokat és meg vagytok győződve róla, hogy a kapott eredmény jó.
Keveseknél láttam szépen papíron levezetett összeadásokat.
A jövőben aki fejben számol és hibásan, az fog kapni egy papíron elvégzendő számolási feladatot.
Most azonban mindenki kap néhányat, de először álljon itt emlékeztetőül, hogy miképpen kell összeadni:

,,Az összeadást többjegyű összeadandók esetén a következő módon végezzük: a számokat úgy írjuk egymás alá, hogy az azonos helyi értékű számjegyek egymás alá kerüljenek.
Az utolsó összeadandót aláhúzzuk. Például:

{\framed[width=\textwidth,frame=off,align=middle,after={\blank}]{
{\framed[frame=off,align={flushright,nothyphenated,verytolerant}]{
\tt 2435\\
247\\
\underbar{+1413}\\
4095\\
}}
}}

Az összeadást a legkisebb helyi értékű jegyenél kezdjük: $3+7+5=15$.
Az eredmény $5$ egyesét leírjuk az egyesek alá, $1$ tizesét pedig a tízesekhez adjuk hozzá: $1+1+4+3=9$.
A százasok összege: $4+2+4=10$, tehát a százasok oszlopa alá $0$-t írunk, és az $1$ ezrest az ezresekhez adjuk hozzá: $1+1+2=4$.
A $4$-et az ezresek alá írjuk. 
A három adott többjegyű szám összege tehát 4095.
Az összeadást felülről lefelé is végezzük el, mert számításunk helyességét így ellenőrizhetjük.''

\startsubject[title=Prímszámok, prímtényezőkre bontás]



\emptylines[1]\noindentation Hosszúhetény, 2018. május 5.

\stopsubject

\stopbodymatter
\stopproduct
\stopproject
