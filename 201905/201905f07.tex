\startproject iras
\mainlanguage[hu]
\setupalign[
  justified,
  nothanging,
  nohz,
  hyphenated,
  morehyphenated,
  tolerant,
]
\setupinterlinespace[height=0.75,depth=0.25]
\setuplayout[
  grid=no,
  location=middle,
]
\setupformulas[align=middle]
%\setupmathalignment[grid=no]

\def\PagenumberingCommand#1{\doifnot\pagenumber1{#1}}
\setuppagenumbering[
  location={footer,middle},
  command=\PagenumberingCommand,
]
\setuppapersize[A4]
\setuplayout[
    backspace=30mm,
    width=150mm,
    topspace=30mm,
    header=0mm,
    footer=15mm,
    footerdistance=0mm,
    bottom=0mm,
    bottomdistance=0mm,
    height=247mm
]

% Betűkészlet
%\setupbodyfont[libertinus,12pt]
\setupbodyfont[12pt]

% Vékony spácium bizonyos karakterek előtt (:;?!)
\definecharacterspacing [magyarpunctuation]
\setupcharacterspacing [magyarpunctuation] ["0021] [left=.1,alternative=1] % ! % strip preceding space(char)
\setupcharacterspacing [magyarpunctuation] ["003A] [left=.1,alternative=1] % : % strip preceding space(char)
\setupcharacterspacing [magyarpunctuation] ["003B] [left=.1,alternative=1] % ; % strip preceding space(char)
\setupcharacterspacing [magyarpunctuation] ["003F] [left=.1,alternative=1] % ? % strip preceding space(char)

% A magyar nyelv beállításai
\startsetups[magyar]
  % Vékony spácium bizonyos karakterek előtt (:;?!)
  \setcharacterspacing[magyarpunctuation]
  \setupindenting[%
    yes,% A bekezdéseket behúzással kezdjük.
    %next,% Az első bekezdés nincs behúzva.
    2em% Közepes méretű (átmeneti megoldás: igazából a mérete 24 cicerós sorig 1 kvirt, nagyobbbnál 2 kvirt kellene legyen -> TENNIALÓ)
  ]
\stopsetups

\setuplanguage[hu][%
  setups=magyar,% Érvényesíti a fent megadott beállításokat.
  spacing=packed% Frenchspacing (Gyurgyák 319. o.: egyenletes szóközök).
                % http://wiki.contextgarden.net/French_spacing).
]

% TENNIVALÓ: csak magyar nyelvre
% Idézetek (Gyurgyák, 86--87. o.).
\definedelimitedtext[quote][location=text]
\setupdelimitedtext[quote:1][
  left={\lowerleftdoubleninequote},
  right={\upperrightdoubleninequote},
  spaceafter=0
]
\setupdelimitedtext[quote:2][
  left={\rightguillemot\nobreak\hskip-.07em},
  right={\kern-0.03em\leftguillemot},
  spaceafter=0
]
\setupdelimitedtext[quote:3][
  left={\upperleftsingleninequote},
  right={\upperrightsingleninequote},
  spaceafter=0
]

\definebodyfontenvironment[default][em=italic]

\defineframedtext[kerdes][align=center,offset=0.5ex,style=italic,width=\dimexpr0.8\dimexpr\makeupwidth]

\defineframedtext[kivonat][offset=0.5ex, frame=off,style=italic,width=\dimexpr0.8\dimexpr\makeupwidth]

\definehead[cim][chapter]
\setuphead[cim][number=no,align=middle,after={},]

\define[2]\sectioncommand{\hbox{#1. #2}}
\setuphead[section][
  sectionsegments=section,
  command=\sectioncommand,
]

\define[2]\feladatcommand{\hbox{Feladat #1 \enskip #2}}
\definehead[feladat][subsection]
\setuphead[feladat][
    number=yes,
    textdistance=0pt,
    alternative=text,
    style=bf,
    commandafter={\enskip},
    command=\feladatcommand,
    sectionsegments=3:100,
    indentnext=no,
    beforesection={\setupindenting[no]},
    aftersection={\setupindenting[yes]},
]

\definebar[mathubar][underbar]  % enélkül esetleg elcsúszik az aláhúzás

%\showgrid[all]



% https://tex.stackexchange.com/a/459615/50554
\def\feladatpont#1{{\unskip\nobreak\hfil\penalty50
  \hskip2em\hbox{}\nobreak\hfil #1\/ 
  \parfillskip=0pt \finalhyphendemerits=0 \par}}

\defineitemgroup[alfeladat]
\setupitemgroup[alfeladat][each][right=),margin=1em,before={},inbetween={},after={},stopper=,style=italic]

\startproduct iras

\useURL[segedanyag][https://github.com/matektankor/segedanyag]

\startbodymatter

\cim{Beadandó dolgozat feladatok}
\startlinealignment[middle]
ötödik osztály, 2019. május, 7/10
\stoplinealignment

\blank[2*big]

\noindentation Kedves tanulók! 
\blank
\noindentation Úgy érzem sikerült megtalálni a közös utat, ami az órák tartalmi részét illeti.
A száraz és szigorú matematikai logika végigjárása helyett a játékos vonalon haladunk, és úgy érzem, hogy ezzel többet tanultatok, több matematikai tudás ragadt meg, mint korábban.

A legutóbbi beadandó dolgozaton rendkívül sokat dolgoztam.
Sajnos erre most nem volt lehetőségem, ráadásul nem adhatok két hetet a beadásra sem, mert szorít minket az év vége.
Ezért ez alkalommal a készségfejlesztés lesz az elsődleges cél, ami egy egyszerűbb, de sajnos kevésbé izgalmas feladatsort fog eredményezni.

A könnyedebb órák ellenére is meg kell tanulni a matematikai módszereket és összefüggéseket.
Ezeken az órákon is bemutattam és kipróbáltunk számos ilyen módszert.
Nem kérek olyasmit tehát, amit nem láttatok, és amit nem próbáltunk ki.

Mindenki kapott segédlapot a mértékegységek átváltásához.
Aki viszont elhagyta, itt tudja pótolni:

\framed[frame=off,align=middle,width=\textwidth,offset=4pt]{
\framed[location=middle,offset=4pt]{\url[segedanyag]}
}

Arra kérlek titeket, hogy először mindenképpen dolgozzatok egyedül, és a szüleiteket vagy más felnőtteket csak akkor vonjatok be a feladatok megoldásába, ha feltétlenül szükséges, és csak annyi ideig, amíg meg nem értitek azt amit egyedül nem sikerült.
Ez a feladat nektek van kiadva.
Többet is tanultok belőle, ha egyedül jöttök rá valamire.


\startsection[title=Természetes törtek]

A természetes törtek teljes megértéséig még nem jutottatok el.
Nem kell aggódni, voltam a hatodikasoknál, és ők sem értik még ezeket teljesen.
Igazából még nem nagyon találkoztam olyannal, aki nyolcadikos koráig eljutott volna oda, hogy teljes könnyedséggel kezelje ezeket, és ismerje az átjárást a természetes és a tizedes törtek, valamint a százalékos alak között.
Habár ezek ismerete év végére már szükséges volna, nem léphetünk túl nagyot, ezért a következő feladatok nem lesznek túl bonyolultak.

A természetes törtek esetén meg kell szokni, hogy nem a felső, hanem az alsó számra (a nevezőre) tekintünk először, ugyanis az mondja meg, hogy mennyibe tört dolgokkal van dolgunk.
Ha a nevező 2, akkor fél részeink vannak, ha 4, akkor negyed részeink, és ha 118, akkor száztizennyolcad részeink.
Minden rész ugyanolyan mint a többi, és együtt 118-an adnak egy egészt, 236-an pedig két egészt.
Ha viszont csak 59 ilyenünk van, akkor csak fél egészünk van.
Azt, hogy mennyi van ezekből az azonos darabkákból, a felső szám (a számláló) mondja meg. Az utolsó esetben, amikor 59 darab 118-ad részünk volt, a számláló tehát 59, a tört pedig $\displaystyle{\fraction{59}{118}}$.

Miért egy fél egész az $\displaystyle{\fraction{59}{118}}$?
Ha a számlálónak és a nevezőnek van közös osztója\footnote{ennek kiderítésére alkalmas a prímtényezős felbontás}, akkor e közös osztóval eloszthatjuk külön-külön a számlálót is és a nevezőt is, és a kapott tört értéke nem változik, azaz továbbra is ugyanazt a dolgot fogja jelenteni, mint az egyszerűsítés előtt.
E műveletet ugyanis egyszerűsítésnek hívjuk.
Vegyük észre, hogy az $\displaystyle{\fraction{59}{118}}$ számlálója is és nevezője is maradék nélkül osztható 59-cel.
Az egyszerűsítés után tehát az $\displaystyle{\fraction{1}{2}}$ törtet kapjuk, ami egy fél egész.

A szmlálót és a nevezőt meg is szorozhatjuk ugyanazzal a számmal, természetesen akkor sem változik a tört értéke.
$\displaystyle{\fraction{1}{2}}$ és $\displaystyle{\fraction{2}{4}}$ tehát ugyanannyi, ahogy $\displaystyle{\fraction{59}{118}}$ is, amikor az $\displaystyle{\fraction{1}{2}}$-et 59-cel bővítjük.
Ez utóbbi művelet neve ugyanis a bővítés.

Ha két tört nevezője azonos, akkor az összeadás és kivonás könnyű velük: csak a számlálókat kell összeadni vagy kivonni.
\startformula \startmathalignment[n=1]
\NC {\fraction{1}{2}}+{\fraction{1}{2}}={\fraction{2}{2}}={\fraction{1}{1}}=1 \NR[+]
\NC {\fraction{1}{2}}+{\fraction{5}{2}}=3 \NR[+]
\NC {\fraction{70}{118}}-{\fraction{11}{118}}={\fraction{59}{118}}={\fraction{1}{2}} \NR[+]
\stopmathalignment \stopformula

Ha viszont a két törtnek nem azonos a nevezője, akkor nem tudjuk azonnal elvégezni az összeadást vagy a kivonást, előbb valamiféleképpen közös nevezőre kell hozni őket.
Úgy kell tehát egyszerűsítenünk vagy bővítenünk egyiket, másikat, vagy mindkettőt, hogy a két tört nevezője azonos legyen.
Ha például a feladat a következő kiszámítása, akkor a számlálókkal nem végezhetjük el a kivonást, mert a nevezők különböznek:
\startformula \startmathalignment[n=1]
\NC {\fraction{3}{2}}-{\fraction{59}{118}} \NR[+]
\stopmathalignment \stopformula
Ha viszont az $\displaystyle{\fraction{59}{118}}$-at egyszerűsítjük $\displaystyle{\fraction{1}{2}}$ formába, akkor máris láthatjuk, hogy
\startformula \startmathalignment[n=1]
\NC {\fraction{3}{2}}-{\fraction{1}{2}}={\fraction{2}{2}}=1. \NR[+]
\stopmathalignment \stopformula
Természetesen bővíthettük volna a $\displaystyle{\fraction{3}{2}}$-et is 59-cel, azaz megszorozzuk a számlálót és a nevezőt is 59-cel.
Ezzel a nevező 118 lesz, a számláló pedig $3\cdot59=177$.
Így
\startformula \startmathalignment[n=1]
\NC {\fraction{177}{118}}-{\fraction{59}{118}}={\fraction{118}{118}}=1. \NR[+]
\stopmathalignment \stopformula

Muszáj volt ide sürítenem az elméletet, mert a külön segédanyagokat nagyon kevesen olvastátok el.

\startfeladat
Ha eddig nem tetted, olvasd el a fenti szöveget a természetes törtekről, és foglald össze legalább 30 szóval.\feladatpont{10 pont}

Volt valami, amit nem értettél belőle? Ha igen, írd le legalább 20 szóval.\feladatpont{5 pont}
\stopfeladat

\startfeladat
Mennyi egészt írnak le a következő tört alakok:\feladatpont{2 pont/db}

{\starttabulate[|Mcw(\dimexpr0.25\textwidth-0.75em\relax)|Mcw(\dimexpr0.25\textwidth-0.75em\relax)|Mcw(\dimexpr0.25\textwidth-0.75em\relax)|Mcw(\dimexpr0.25\textwidth-0.75em\relax)|][distance=big]
 \NC {\fraction{3}{3}} \NC {\fraction{14}{7}} \NC {\fraction{12345}{12345}} \NC {\fraction{10}{1}}\NR \TB[line]
 \NC {\fraction{400}{2}} \NC {\fraction{1}{1}} \NC {\fraction{125}{5}} \NC {\fraction{500}{100}}\NR \TB[line]
 \NC {\fraction{99}{9}} \NC {\fraction{990}{9}} \NC {\fraction{999}{9}} \NC {\fraction{1008}{9}}\NR \TB[line]
\stoptabulate
}  
\stopfeladat

\startfeladat
Egyszerűsítsd a következő törteket:\feladatpont{2 pont/db}

{\starttabulate[|Mcw(\dimexpr0.25\textwidth-0.75em\relax)|Mcw(\dimexpr0.25\textwidth-0.75em\relax)|Mcw(\dimexpr0.25\textwidth-0.75em\relax)|Mcw(\dimexpr0.25\textwidth-0.75em\relax)|]
  \NC {\fraction{18}{14}} \NC {\fraction{16}{16}} \NC {\fraction{144}{192}} \NC {\fraction{14}{18}}\NR \TB[line]
 \NC {\fraction{651}{372}} \NC {\fraction{230}{276}} \NC {\fraction{9}{63}} \NC {\fraction{12}{16}}\NR \TB[line]
 \NC {\fraction{1}{8}} \NC {\fraction{7}{7}} \NC {\fraction{3}{3}} \NC {\fraction{939}{939}}\NR \TB[line]

\stoptabulate
}  
\stopfeladat

\startfeladat
Bővítsd a törteket a 2-től 10-ig terjedő természetes számokkal. Minden számot csak egyszer használhatsz, de te döntheted el, hogy melyik törtet bővíted 2-vel, melyiket 3-mal, melyiket 4-gyel és így tovább:\feladatpont{2 pont/db}

{\starttabulate[|Mcw(\dimexpr0.33\textwidth-0.67em\relax)|Mcw(\dimexpr0.34\textwidth-0.66em\relax)|Mcw(\dimexpr0.33\textwidth-0.67em\relax)|]
  \NC {\fraction{7}{5}} \NC {\fraction{3}{53}} \NC {\fraction{13}{2}}\NR \TB[line]
 \NC {\fraction{9}{72}} \NC {\fraction{12}{2}} \NC {\fraction{17}{76}}\NR \TB[line]
 \NC {\fraction{50}{34}} \NC {\fraction{32}{27}} \NC {\fraction{1}{6}}\NR \TB[line]

\stoptabulate
}  
\stopfeladat

\startfeladat
Végezd el az alábbi műveleteket:\feladatpont{2 pont/db}

{\starttabulate[|Mcw(\dimexpr0.33\textwidth-0.67em\relax)|Mcw(\dimexpr0.34\textwidth-0.66em\relax)|Mcw(\dimexpr0.33\textwidth-0.67em\relax)|]
%{\starttabulate[|Mcw(\dimexpr0.5\textwidth-0.5em\relax)|Mcw(\dimexpr0.5\textwidth-0.5em\relax)|]
  \NC {\fraction{107}{4}}-{\fraction{4}{4}} \NC {\fraction{4}{69}}-{\fraction{3}{69}} \NC {\fraction{3}{51}}+{\fraction{144}{51}}\NR \TB[line]
 \NC {\fraction{75}{26}}-{\fraction{5}{26}} \NC {\fraction{35}{6}}+{\fraction{2}{6}} \NC {\fraction{83}{5}}+{\fraction{8}{5}}\NR \TB[line]
 \NC {\fraction{52}{2}}+{\fraction{71}{2}} \NC {\fraction{98}{4}}+{\fraction{510}{4}} \NC {\fraction{1}{7}}+{\fraction{20}{7}}\NR \TB[line]
 \NC {\fraction{50}{26}}+{\fraction{604}{26}} \NC {\fraction{64}{47}}-{\fraction{48}{47}} \NC {\fraction{20}{4}}+{\fraction{32}{4}}\NR \TB[line]

\stoptabulate
}  
\stopfeladat

\startfeladat
Végezd el az alábbi műveleteket:\feladatpont{4 pont/db}

{\starttabulate[|Mcw(\dimexpr0.33\textwidth-0.67em\relax)|Mcw(\dimexpr0.34\textwidth-0.66em\relax)|Mcw(\dimexpr0.33\textwidth-0.67em\relax)|]
%{\starttabulate[|Mcw(\dimexpr0.5\textwidth-0.5em\relax)|Mcw(\dimexpr0.5\textwidth-0.5em\relax)|]
  \NC {\fraction{7}{91}}-{\fraction{1}{253}} \NC {\fraction{9}{12}}-{\fraction{5}{212}} \NC {\fraction{97}{195}}+{\fraction{6}{24}}\NR \TB[line]
 \NC {\fraction{79}{393}}+{\fraction{790}{1}} \NC {\fraction{71}{7}}-{\fraction{3}{4}} \NC {\fraction{881}{47}}-{\fraction{264}{524}}\NR \TB[line]
 \NC {\fraction{729}{2}}+{\fraction{1}{24}} \NC {\fraction{7}{6}}-{\fraction{1}{5}} \NC {\fraction{78}{4}}-{\fraction{27}{5}}\NR \TB[line]
 \NC {\fraction{90}{13}}-{\fraction{3}{333}} \NC {\fraction{84}{2}}+{\fraction{92}{862}} \NC {\fraction{5}{9}}-{\fraction{3}{79}}\NR \TB[line]

\stoptabulate
}  
\stopfeladat


\startsection[title=Mértékek, mértékegységek és átváltásuk]

\startfeladat
Fejezd ki az alapegységben és jelezd, hogy milyen mértékről van szó:\feladatpont{$2+1$ pont/db}

{\starttabulate[|cw(\dimexpr0.33\textwidth-0.67em\relax)|cw(\dimexpr0.34\textwidth-0.66em\relax)|cw(\dimexpr0.33\textwidth-0.67em\relax)|]
  \NC 5000\,cm \NC 1\,m \NC 1\,d\NR 
 \NC 943\,ha \NC 659000\,l \NC 9136\,m\high{3}\NR 
 \NC 23000\,mm \NC 8\,dkg \NC 92000\,g\NR 
 \NC 5\,km \NC 89\,s \NC 22\,h\NR 
 \NC 70000000\,mg \NC 6\,kg \NC 20\,min\NR 
 \NC 30\,m\high{2} \NC 987\,t \NC 30\,dm\NR 

\stoptabulate
}  
\stopfeladat

\startfeladat
Fejezd ki a megadott egységban és jelezd, hogy milyen mértékről van szó:\feladatpont{$4+1$ pont/db}

{\starttabulate[|Mcw(0.5\textwidth)||Mcw(0.5\textwidth)|][unit=0pt]
  \NC \text{80\,m\high{2}} = \text{\dots\,ha} \NC \text{85\,mm} = \text{\dots\,dm}\NR 
 \NC \text{658\,ml} = \text{\dots\,hl} \NC \text{715\,cm\high{2}} = \text{\dots\,dm\high{2}}\NR 
 \NC \text{33\,cl} = \text{\dots\,l} \NC \text{5\,dm\high{3}} = \text{\dots\,m\high{3}}\NR 
 \NC \text{6583\,cm} = \text{\dots\,m} \NC \text{5\,g} = \text{\dots\,t}\NR 
 \NC \text{486\,cm} = \text{\dots\,m} \NC \text{889\,min} = \text{\dots\,d}\NR 
 \NC \text{44\,s} = \text{\dots\,h} \NC \text{9401\,cm\high{3}} = \text{\dots\,dl}\NR 
 \NC \text{5008\,mm\high{3}} = \text{\dots\,km\high{3}} \NC \text{9\,mg} = \text{\dots\,t}\NR 
 \NC \text{59\,ha} = \text{\dots\,km\high{2}} \NC \text{3\,s} = \text{\dots\,h}\NR 
 \NC \text{71\,dm} = \text{\dots\,km} \NC \text{873\,mm} = \text{\dots\,km}\NR 
 \NC \text{20\,dl} = \text{\dots\,l} \NC \text{2143\,mm\high{3}} = \text{\dots\,l}\NR 

\stoptabulate
}  
\stopfeladat



\emptylines[1]\noindentation Hosszúhetény, 2019{.} május 27.

\stopsubject

\stopbodymatter
\stopproduct
\stopproject
